\section{Sets, Functions, and Integers}
\subsection{Sets}

\subsubsection{Exercise 5}
When constructing a subset, each element in the set can either be in or out (2 choices).
Hence, $2^n$.

\subsubsection{Exercise 6}
There are $n$ choices for the first element, $n-1$ choices for the second element,
and so on up to $n-m$, hence dividing $n!$ by $(n-m)!$. The order of these $m$ selected
elements doesn't matter, hence the division by $m!$.

\subsection{Functions}

\subsubsection{Exercise 2}
$h_g \circ h_f$, where $h$ corresponds to left-inverse.

\subsubsection{Exercise 3}
Let $f:A \to B$ and $g:B \to C$ be surjections. Then  $g \circ f$ is surjective
since $\exists x \in B$ such that $g(x) = y \quad \forall y \in C$, and 
$\exists x' \in A$ such that $f(x') = x \quad \forall x \in B$
(from the surjectivity of $f$ and $g$). Proving injectivity follows similarly.

\subsubsection{Exercise 4}
The reverse direction follows from Exercise 3. If $f \circ g$ is injective and $g$ is not,
we could choose two elements from the domain of $g$ that map to the same element in 
the domain of $f$ (contradiction). Surjectivity is a similar argument.

\subsubsection{Exercise 5}
$f$ has no right inverse since it is not surjective.
There are infinitely many left inverses of $f$, two possibilities are 
mapping to square roots when possible and to 1 or 2 otherwise.

\subsubsection{Exercise 6}
Apply the left inverse of $f$.

\subsubsection{Exercise 7}
When surjective, use right inverse.

\subsubsection{Exercise 8}
Define $h$ such that $h(y) = x$ if $\exists x \in S \: | f(x) = y$, and $h(y) = x'$ otherwise
(axiom of choice necessary for choosing $x$). If $f$ is injective, there will only be
one choice of $x$, and if $f$ is surjective, there will be some $x$ for every $y$.

\subsubsection{Exercise 9}
Unique right inverse indicates that every element in the range has
only one choice to map back to in the domain, implying injectivity.

\subsubsection{Exercise 10}
If $g$ is a bijection, then we can define $f$ such that $f(y) = x$ 
where $g(x) = y$. $f$ is then a two-sided inverse.
If  $f$ is a two-sided inverse of $g$, then every element of $T$ maps to a
unique element of $S$ (from left inverse) and vice versa. Hence $g$ is
a bijection.

\subsubsection{Exercise 11}
Following the hint, we can see that $f: U \to \mathcal{F}$ is surjective since
$S \in \mathcal{F} \implies S \neq \emptyset \implies \exists u \in S \implies u \in U \implies f(u) = S$.
The existence of the right inverse then gives us the axiom of choice.
