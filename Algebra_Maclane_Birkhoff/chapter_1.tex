\section{Sets, Functions, and Integers}
\subsection{Sets}

\subsubsection{Exercise 5}
When constructing a subset, each element in the set can either be in or out (2 choices).
Hence, $2^n$.

\subsubsection{Exercise 6}
There are $n$ choices for the first element, $n-1$ choices for the second element,
and so on up to $n-m$, hence dividing $n!$ by $(n-m)!$. The order of these $m$ selected
elements doesn't matter, hence the division by $m!$.

\subsection{Functions}

\subsubsection{Exercise 2}
$h_g \circ h_f$, where $h$ corresponds to left-inverse.

\subsubsection{Exercise 3}
Let $f:A \to B$ and $g:B \to C$ be surjections. Then  $g \circ f$ is surjective
since $\exists x \in B$ such that $g(x) = y \quad \forall y \in C$, and 
$\exists x' \in A$ such that $f(x') = x \quad \forall x \in B$
(from the surjectivity of $f$ and $g$). Proving injectivity follows similarly.

\subsubsection{Exercise 4}
The reverse direction follows from Exercise 3. If $f \circ g$ is injective and $g$ is not,
we could choose two elements from the domain of $g$ that map to the same element in 
the domain of $f$ (contradiction). Surjectivity is a similar argument.

\subsubsection{Exercise 5}
$f$ has no right inverse since it is not surjective.
There are infinitely many left inverses of $f$, two possibilities are 
mapping to square roots when possible and to 1 or 2 otherwise.

\subsubsection{Exercise 6}
Apply the left inverse of $f$.

\subsubsection{Exercise 7}
When surjective, use right inverse.

\subsubsection{Exercise 8}
Define $h$ such that $h(y) = x$ if $\exists x \in S \: | \: f(x) = y$, and $h(y) = x'$ otherwise
(axiom of choice necessary for choosing $x$). If $f$ is injective, there will only be
one choice of $x$, and if $f$ is surjective, there will be some $x$ for every $y$.

\subsubsection{Exercise 9}
Unique right inverse indicates that every element in the range has
only one choice to map back to in the domain, implying injectivity.

\subsubsection{Exercise 10}
If $g$ is a bijection, then we can define $f$ such that $f(y) = x$ 
where $g(x) = y$. $f$ is then a two-sided inverse.
If  $f$ is a two-sided inverse of $g$, then every element of $T$ maps to a
unique element of $S$ (from left inverse) and vice versa. Hence $g$ is
a bijection.

\subsubsection{Exercise 11}
Following the hint, we can see that $f: U \to \mathcal{F}$ is surjective since
$S \in \mathcal{F} \implies S \neq \emptyset \implies \exists u \in S \implies u \in U \implies f(u) = S$.
The existence of the right inverse then gives us the axiom of choice.

\subsection{Relations and Binary Operations}

\subsubsection{Exercise 2}
Symmetry + transitivity imply circularity. 
For the other direction, we have 
$xRy, \: yRy \implies yRx$, which gives both symmetry and transitivity.

\subsubsection{Exercise 3}
This only implies reflexivity for the elements $x, y \in X \: | \: (x, y) \in R$, not $\forall x \in X$.

\subsubsection{Exercise 4}
If $R$ is transitive $T = R$. Otherwise, start with $T = R$ and add $(x, z)$ to $T$
whenever  $(x, y), (y, z) \in R$. Repeat this process until there are no more pairs to add.

\subsubsection{Exercise 5}
Let $R \subset X \times Y, \: S \subset Y \times Z, \: T \subset Z \times A$.
\begin{align*}
        x R \circ (S \circ T) a &\implies \exists y \in Y \: | \: xRy, y (S \circ T) a \\
        &\implies \exists z \in Z \: | \: ySz, zTa \\
        &\implies x (R \circ S) z \\
        &\implies x (R \circ S) \circ T a
\end{align*}

\subsubsection{Exercise 6}
Let $R \subset X \times Y, \: S \subset Y \times Z$.
\begin{align*}
        z (R \circ S)^{\smile} x &\implies x (R \circ S) z \\
                                 &\implies \exists y \in Y \: | \: xRy, ySz \\
                                 &\implies yR^{\smile} x, \:  zS^{\smile} y \\
                                 &\implies z (S^{\smile} \circ R^{\smile}) x
\end{align*}

\subsubsection{Exercise 7}
\begin{align*}
        (x, z) \in G(g \circ f) &\implies \exists y \in Y \: | \: g(y) = z, \: f(x) = y \\
                                &\implies (x, y) \in G(f), \: (y, z) \in G(g) \\
                                &\implies (x, z) \in G(f) \circ G(g)
\end{align*}

\subsubsection{Exercise 9} 
\begin{align*}
        (x, y) \in G(f) &\implies \forall x \in X, \: \exists y \in Y \: | \: f(x) = y \\
                        &\implies \forall x \in X, \: (x, x) \in G(f) \circ G^{\smile} (f) \\
                        &\text{and} \quad \forall y \in \text{Im} f, \: (y, y) \in G^{\smile} (f) \circ G(f)
\end{align*}

\subsubsection{Exercise 10}
\begin{align*}
        &x \square y = u \square (x \square y) = (u \square y) \square x = y \square x \\
        &x \square (y \square z) = x \square (z \square y) = (x \square y) \square z
\end{align*}

\subsection{The Natural Numbers}

\subsubsection{Exercise 1}
$f^0 = 1_X$ is trivially an injection. Suppose $f^n$ is an injection for some $n \in \mathbb{N}$.
Then $f^{\sigma(n)} = f \circ f^n$ is a composition of injections and we are done.

\subsubsection{Exercise 2}
Same thing as Exercise 1.

\subsubsection{Exercise 3}
We have that $\sigma^0(0) = 0$. Now assuming  $\sigma^n(0) = n$ for some $n \in \mathbb{N}$,
we have $\sigma^{\sigma(n)}(0) = \sigma \circ \sigma^n(0) = \sigma(n) = n + 1$.

\subsubsection{Exercise 6}
We can take $\sigma^{-1}(n) = n - 1$ for $n > 0$ and $\sigma^{-1}(0) = 0, 1, 2$
to get 3 different left inverses.

\subsubsection{Exercise 8}
Let $n \in U$ if the elements in all sets of size $n$ are equal.
Since we can construct a set with two different elements, we have that $n = 1$ does
not imply $\sigma(n) \in U$, and the induction axiom cannot be applied to $U$.

\subsubsection{Exercise 9}
(Property I, Property II): Take $X = \mathbb{N}$ and $\sigma(x) = x^2 + 1$.

(Property I, Property III): Let $X = \{0, 1\}$ and let $\sigma(0) = 1, \: \sigma(1) = 0$.
Then $\sigma$ is clearly injective, and any subset of $X$ that contains 0 and $\sigma(0)$ 
is all of $X$.

(Property II, Property III): Again take $X = \{0, 1\}$, but this time let $\sigma(0) = \sigma(1) = 1$.

\subsection{Addition and Multiplication}

\subsubsection{Exercise 1}
\begin{align*}
        n = 0 &:  \: (f^m)^0 = 1 = f^0 = f^{(\sigma^m)^0 (0)} = f^{m 0} \\
        \text{Assume n} &: \: (f^m)^{(\sigma(n))} = f^m \circ f^{mn} = f^{m(n+1)}
\end{align*}

\subsubsection{Exercise 2}
(a) $mn = (\sigma^m)^n (0) = \sigma^{mn} (0) = \sigma^{nm} (0) = nm$.

(b) $\sigma(m) (n + n') = (\sigma^{\sigma(m)})^{n + n'}(0) = (\sigma^{\sigma(m)})^n(0) + (\sigma^{\sigma(m)})^{n'}(0)$.

\subsubsection{Exercise 3}
(a) To obtain a valid $\tau$, simply permute the first few mappings of $\sigma$.
For example, $\tau(0) = 2, \tau(1) = 3, \tau(2) = 1, n \geq 3 : \: \tau(n) = n + 1$.

(b) Suppose $\tau$ satisfies Peano. Then we can let $\beta(0) = 0$ and 
$\beta(n) = \tau(\beta(n - 1))  \: \forall n > 0$. $\beta$ is a bijection since
$\tau$ is injective and maps to all of $\mathbb{N} / \{0\}$.
Furthermore, $\beta \sigma (n) = \beta (n+1) = \tau \beta (n)$.

\subsubsection{Exercise 4}
(a)
\begin{align*}
        \phi(n) = m &\implies \sigma(\phi(n)) = m + 1 \\
                    &\implies \phi(\sigma(n)) = \phi(n + 1) = m + 1
\end{align*}
Thus, once we fix $\phi(0)$, we fix the rest of $\phi$.

(b) There is only one choice of $\tau$ which satisfies Peano's Postulates:
$\tau(0) = 1$ with $\tau$ satisfying the relation indicated in (a). 
This is exactly the successor function $\sigma$.

\subsubsection{Exercise 6}
$k + n = \sigma^n(k) = \sigma^n(m) \implies k = m$ since a composition of injections
is an injection.

\subsection{Inequalities}

\subsubsection{Exercise 1}
Since  $x = x$ we have reflexivity of $\leq$.
Since $x \leq y \implies x + a = y$ and $y \leq z \implies y + b = z$,
we have $x + a + b = z$ giving transitivity.

\subsubsection{Exercise 2}
 \begin{align*}
         m < n &\implies m + x = n \\
               &\implies m + x + k = n + k \\
               &\implies m + k < n + k
\end{align*}
Multiplication is also isotonic since it's just iterated addition.

\subsubsection{Exercise 3}
Suppose $0 \in U, \: n \in U \implies \sigma(n) \in U$ and $U \neq \mathbb{N}$.
Then from well-ordering, we have that $\mathbb{N} / U$ has a first element $f$ 
such that $m < f \implies m \in U$. However, this gives us that
$\exists m \in  U \: | \: \sigma(m) = f$ which leads to a contradiction.

\subsubsection{Exercise 4}
Suppose $S$ is well-ordered with first element $f$ but $U \subset S$ is not.
Then $V \subset U \: | \: V \neq \emptyset$ and $V$ has no first element.
However, since $V \subset S$, we have a contradiction, since well-ordering
implies that every subset of $S$ has a first element.

\subsubsection{Exercise 6}
The subset consisting of that infinite descending sequence would contain no
first element.

\subsection{The Integers}

\subsubsection{Exercise 1}
Let $u = sdu + u_0$ and let $v = sdv + v_0$.
\begin{align*}
        uv &= (sdu) (sdv) + (sdu) (v_0) + (u_0) (sdv) + u_0 v_0 \\ 
        d(uv) &= d((sdu) (sdv)) + 0 + 0 + 0 \\
              &= (du) (dv)
\end{align*}

\subsubsection{Exercise 3}
Follows from the steps of lemma, since we have that $du \oplus' dv = d(u + v) = d(sdu + sdv) = du \oplus dv$.

\subsubsection{Exercise 4}
Suppose $a \oplus x_1 = a \oplus x_2$. Then $a' \oplus (a \oplus x_1) = a' \oplus (a \oplus x_2)$,
which gives $x_1 = x_2$.

\subsubsection{Exercise 5}
Same logic as Exercise 3, except using the result of Exercise 1.

\subsection{The Integers Modulo $n$}

\subsubsection{Exercise 3}
\begin{align*}
        h - k \in n\mathbb{Z},\:  r - s \in n\mathbb{Z} &\implies (h - k) + (r - s) \in n\mathbb{Z} \\
                                                     &\implies (h + r) - (k + s) \in n\mathbb{Z} \\
        h (r - s) \in n\mathbb{Z},\: s (h - k) \in n\mathbb{Z} &\implies h (r - s) + s (h - k) \in n\mathbb{Z} \\
                                                               &\implies hr - ks \in n\mathbb{Z}
\end{align*}

\subsubsection{Exercise 4}
Just check the squares of $0, ..., 7$ mod 8 to get the desired result.

\subsubsection{Exercise 5}
7 cannot be decomposed into a sum of 3 integers from the set $\{0, 1, 4\}$.

\subsubsection{Exercise 6}
One of the three consecutive integers must be divisible by 3; let the remainder of this integer mod 9 be
$k$. Then, WLOG, we can let the other two integers be $k - 1$ and $k + 1$ mod 9. We then have that
$(k - 1)^3 + k^3 + (k + 1)^3 = 3k^3 + 6k$, which is divisible by 9 since $k$ is divisible by 3.

\subsection{Equivalence Relations and Quotient Sets}

\subsubsection{Exercise 1}
The quotient $T / S$ consists of the set of all possible equivalence classes of triangles
based on the relation of triangle similarity. Thus, each element of $T / S$ corresponds to a
different kind of triangle similarity, or ``shape''.

\subsubsection{Exercise 2}
$p \times p$ is an equivalence relation on $\mathbb{Z} \times \mathbb{Z}$. Furthermore,
$(p \times p) (x, y) = (p \times p) (x', y') \implies p(x + y) = p(x' + y')$.
Then by Theorem 19, we can define addition of cosets of two integers as the function
that commutes with the coset of the sum of the integers.

\subsubsection{Exercise 3}
Reflexivity and symmetry are clear; transitivity follows from the fact that if  $(x_1, y_1)E(x_2,y_2), \: 
(x_2, y_2)E(x_3, y_3)$, then $x_3 - x_1 = x_3 - x_2 + x_2 - x_1$ which is the sum of two integers and
therefore an integer. 

\subsection{Morphisms}

\subsubsection{Exercise 1}
The additive endomorphisms of $\mathbb{Z}$ are completely determined by the value they map 1 to.
Thus, they are all functions of the form $f(z) = c z$ for some constant $c \in \mathbb{Z}$.

\subsubsection{Exercise 2}
Every additive morphism from $\mathbb{Z}_n$ to $\mathbb{Z}_m$ is of the form $f(z) = p_m (c z)$ 
where $p_m: \mathbb{Z} \to \mathbb{Z}_m$ maps elements of $\mathbb{Z}$ to their remainders mod $m$  
and $c \in  \mathbb{Z}_m$.

\subsubsection{Exercise 3}
Follows the structure indicated in Exercise 2.

\subsubsection{Exercise 4}
Each rotation of the square can be decomposed into clockwise rotations. If we label the vertices of 
the square as $0, 1, 2, 3$, then a clockwise rotation can be thought of as adding 1 mod 4. Thus,
the isomorphisms between $(\mathbb{Z}_4, +)$ and $(Q, \circ)$ are exactly the additive isomorphisms
between $\mathbb{Z}_4$ and itself. There are only 2 such isomorphisms: $f(1) = 1$ and $f(1) = 3$.

\subsubsection{Exercise 5}
Follows from left inverse for injectivity and right inverse for surjectivity.

\subsubsection{Exercise 7}
Any morphism $f: (\mathbb{R}, \times) \to (\mathbb{R}, +)$ satisfies
\begin{align*}
        f(1 * 1) &= f(1) + f(1) \implies f(1) = 0 \\
        f(0 * 0) &= f(0) + f(0) \implies f(0) = 0
\end{align*}
Which means $f$ cannot be an isomorphism.

\subsection{Semigroups and Monoids}

\subsubsection{Exercise 1}
If $u$ and $u'$ are both units, then $u \square u' = u' = u$.

\subsubsection{Exercise 2}
The terms $a_1, ..., a_m$ and $a_{m+1}, ..., a_{m+n}$ together give $a_1, ..., a_{m + n}$.

\subsubsection{Exercise 3}
As stated in the text, follows from induction on $n$ (the proofs can be found in previous sections).

\subsubsection{Exercise 4}
Due to commutativity, we can rearrange the terms in the double sum as we like, thereby allowing us
to swap sums.

\subsubsection{Exercise 5}
Let $f: (\mathbb{N}, +) \to (\mathbb{N}, \times)$ be such that $f(n) = 0 \:  \forall n \in \mathbb{N}$.
Then $f$ is a morphism that does not map the additive unit 0 to the multiplicative unit 1.
