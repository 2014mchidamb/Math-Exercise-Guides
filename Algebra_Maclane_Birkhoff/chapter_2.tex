\section{Groups}

\subsection{Groups and Symmetry}

\subsubsection{Exercise 2}
Map each element $x \in \mathbb{Z}_6$ to the pair $(p_2(x), p_3(x))$.
This is an isomorphism, since the projections $\mathbb{Z}_6 \to \mathbb{Z}_3$ 
and $\mathbb{Z}_6 \to \mathbb{Z}_3$ are both group morphisms, and the mapping
itself is a bijection.

\subsubsection{Exercise 3}
To see that there is no isomorphism $f: \mathbb{Z}_4 \to \mathbb{Z}_2 \times \mathbb{Z}_2$, 
consider $f(1)$ and $f(3)$. We have that $f(0) = f(1 + 3) = f(1) + f(3)$ which is not
possible since $f(0) = (0, 0)$ (has to be the case since $f(x) = f(0) + f(x)$ ).

Rotations do not preserve symmetry for rectangles, since distances between adjacent
vertices change. The only transformations that preserve symmetry are reflections
across the vertical and horizontal axes, giving 4 possible transformations. We can
then map $(0, 0)$ to the identity, $(0, 1)$ to a vertical reflection, $(1, 0)$ to
a horizontal reflection, and $(1, 1)$ to a vertical + horizontal reflection.

\subsubsection{Exercise 4}

\subsubsection{Exercise 5}

\subsubsection{Exercise 6}

\subsubsection{Exercise 10}
The set of these permutations has identity  $(1, 0)$, and any permutation $(a, b)$ has
inverse $(\frac{1}{a}, -\frac{b}{a})$. Furthermore, $(a_2, b_2) \circ (a_1, b_1) = (a_1 a_2, a_2 b_1 + b_2)$,
which is associative since multiplication and addition are both associative.

\subsubsection{Exercise 11}
(a) To show that the given function is a permutation on $\mathbb{R}\cup{\infty}$, we need to show that it is a
bijection from $\mathbb{R}\cup{\infty} \to \mathbb{R}\cup{\infty}$.
Suppose $f(x_1) = f(x_2)$. Then
\begin{align*}
        \frac{ax_1 + b}{cx_1 + d} &= \frac{ax_2 + b}{cx_2 + d} \\
        (ad - bc)x_1 &= (ad - bc)x_2 \implies x_1 = x_2
\end{align*}
So $f$ is an injection from $\mathbb{R}\cup{\infty} \to \mathbb{R}\cup{\infty}$. Furthermore, if
we set $f(x) = y$, we can solve for $x$, which gives us that $f$ is also a surjection.

(b) I'm sure an inverse can be found, but it's tedious... Associativity then follows again from associativity
of multiplication and addition.

\subsubsection{Exercise 12}

\subsubsection{Exercise 13}
(a) Any automorphism of $\mathbb{Z}_3$ has to fix 0. Thus, the only two automorphisms are the identity
and the automorphism that swaps 1 and 2.

(b) Fixing $(0, 0)$, we see that we can permute the remaining three elements as we want, giving
the isomorphism to $S_3$.

(c) 

\subsection{Rules of Calculation}

\subsubsection{Exercise 1}
(a) Multiply by inverse and use associativity.

(b) Associativity.

(c) Associativity and then inverse of product.

\subsubsection{Exercise 2}
Multiply by $a^{-1}$.

\subsubsection{Exercise 3}
Since the unit is its own inverse, we're left with $2n - 1$ elements that need to be paired with one another.
Since $2n - 1$ is odd, we have that one of the elements must be its own inverse.

\subsubsection{Exercise 4}
Any group with 3 elements must be of the form $1, a, a^{-1}$. Thus, each of these groups is clearly
isomorphic to the others.

\subsubsection{Exercise 5}
I struggled to untie the ideas of cancellation and inverse, so I ended up looking up a hint for this one. To
see that an infinite set with cancellation does not need to be a group, consider $(\mathbb{N}, +)$.
This is a monoid that was proven to have cancellation in chapter 1, but does not contain inverses.

For the case of a finite set $G$, we can use the fact that $f(x) = ax$ is an injection for any $a \in G$, 
since $ax = ay \implies x = y$ by cancellation. Since $G$ is finite, $f$ is also a surjection. Therefore,
$\exists a \: | \: ax = 1$ which gives us that there is a left inverse. Applying the same logic using
$f(x) = xa$ gives a right inverse, which completes the proof since these inverses must be equal.

\subsubsection{Exercise 6}
Left cancellation is possible due to left inverse and left unit.
Furthermore, $uu = u \implies (a' a) u = a' a \implies a u = a$ by left cancellation, indicating that 
$u$ is also a right unit. Then we have that $u a' = a' u \implies a' a a' = a' u \implies a a' = u$, 
and $a'$ is also a right inverse. This proves that $X$ is a group.

\subsubsection{Exercise 7}
We proceed as directed in the hint. Since the equation $ua = a$ has solution $u$, and any $b$ 
can be written as $b = ay$, we have $u b = u (ay) = ay = b$. Thus, $u$ is a left unit. Since
the equation $a'a = u$ also has a solution $a'$, we are done by Exercise 6.

\subsubsection{Exercise 10} 
Since each element of $G$ has a unique inverse, $f(a) = a^{-1}$ is a bijection. Additionally,
$f(ab) = (ab)^{-1} = b^{-1} a^{-1} = f(b) f(a) = f(a) \square^{\text{op}} f(b)$.

\subsubsection{Exercise 11}
Associativity of $\square$ immediately follows from the associativity of $G$ 's binary operation and
the fact that $p$ is a morphism. Additionally, since $p$ is an epimorphism, $\forall x, \: \exists g
\: | \: x = p(g)$. Since $u g = g u$, $p(u)$ is then the unit for $X$. Similarly, $p(g')$ is the
inverse of $x$, thus making $X$ a group.

\subsubsection{Exercise 12}
$b b_R = u \implies b_L b b_R = b_L \implies b_R b = b_L b = u$.

\subsection{Cyclic Groups}
We first show that $\mathbb{Z}_n$ is generated only by those $c$ that are coprime to $n$. If 
$c$ is coprime to $n$, then $ac = 0$ only when $a = n$ since $c$ and  $n$ share no prime factors.
Thus, the subgroup generated by $c$ has order $n$ and is therefore all of $\mathbb{Z}_n$. Similarly, 
if $c$ is a generator of $\mathbb{Z}_n$, then $c$ has order $n$ and must therefore be coprime to $n$.

\subsubsection{Exercise 1}
The only possible generators are 1 and 5, since those are the only elements of $\mathbb{Z}_6$ 
that are coprime to 6.

\subsubsection{Exercise 2}
The endomorphisms of $\mathbb{Z}_n$ are completely determined by the mapping of 1, so there are
only $n$ such endomorphisms.

\subsubsection{Exercise 3}
5 is prime, so all elements of $\mathbb{Z}_5$ other than 0 are coprime to it. 

\subsubsection{Exercise 4}
14 has 6 positive integers less than it that are coprime to it (3, 5, 7, 9, 11, 13).

\subsubsection{Exercise 5}
The two generators of $\mathbb{Z}$ are $1$ and $-1$, as elements of $\mathbb{Z}$ can be written as $-m$ or $m$.

\subsubsection{Exercise 7}
If $G$ is abelian, then $(g_1 g_2)^m$ can be rearranged to $g_1^m g_2^m$. If $(g_1 g_2)^m = g_1^m g_2^m$.
The reverse direction follows from the $m = 2$ case, $g_1 g_2 g_1 g_2 = g_1^2 g_2^2$.

\subsubsection{Exercise 8}
$(g_1 g_2) (g_1 g_2) = 1 \implies g_1 g_2 = g_2 g_1$.

\subsubsection{Exercise 9}
The automorphisms are all determined by the mappings of the generators; the isomorphisms follow from the
number of generators of each group.

\subsubsection{Exercise 10}

\subsection{Subgroups}

\subsubsection{Exercise 1}
The subgroup mapping a given diagonal to itself consists of $\{1, R^3, D, D'\}$, where
$R^3$ is 3 clockwise rotations, $D$ is reflection across the given diagonal, and $D'$ is
reflection across the diagonal perpendicular to the given. Mapping those elements to
$\{(0, 0), (1, 1), (0, 1), (1, 0)\}$ (in order) is an isomorphism.

\subsubsection{Exercise 4}
If $S$ is closed under product and inverse, then it contains the identity and is thus a subgroup.

\subsubsection{Exercise 5}
We have that $(s, t) (t, s) = st^{-1} ts^{-1} $, so $S$ contains the identity. Then $(1, s) = s^{-1}$ and
$(s, t^{-1}) = st$, so $S$ is closed under products and inverses as well, thus making it a subgroup.

\subsubsection{Exercise 6}
(a) The identity has order 1. Additionally, if $a$ has finite order, so does $a^{-1}$. Finally,
$a^n = 1, \: b^k = 1 \implies (ab)^{nk} = a^{nk} b^{nk} = 1$.

(b) If non-abelian, we do not necessarily have $(ab)^{nk} = a^{nk} b^{nk}$.

\subsubsection{Exercise 7}
If $G$ has no proper subgroups, then it is generated by all of its non-identity elements. 
This is only possible if $G$ has order 1 (vacuously true), or if $G$ is a cyclic group of
prime order (as was shown in the beginning of the previous section).

\subsubsection{Exercise 8}
(a) If $a$ has order $n$, so does $a^{-1}$. Additionally, $(ab)^n = a^n b^n = 1$, making all elements that
satisfy $a^n = 1$ a subgroup of $A$. To see that this is not true for non-abelian groups, consider $S_3$.
The elements $(1 2)$ and $(2 3)$ are both of order 2, but $(1 2) (2 3) = (1 2 3)$ is of order 3.

(b) That the $n^{\text{th}}$ powers form a subgroup follows from $a^n a^{-n} = 1$ and $a^n b^n = (ab)^{n}$.

\subsubsection{Exercise 9}
If $T$ is a submonoid of $S$, then $i: T \to S$ is a morphism of monoids, so $T$ must necessarily be
closed under products and identity. For the reverse direction, if $T$ is closed under products and
identity, then the insertion $i$ is a morphism of monoids and $T$ is a submonoid of $S$.

\subsection{Defining Relations}

\subsubsection{Exercise 3}
The subgroup of rotations is isomorphic to $\mathbb{Z}_5$, so each element other than the identity has order 5.
The element $D$ has order 2. Furthermore, we have from the generator relations that
\begin{align*}
        DR = R^{n-1}D \implies DR^i = R^{n-1} DR^{i-1} = DR^i = R^{i(n - 1)}D = R^{n - i}D
\end{align*}
So all elements of the form $DR^i$ also have order 2.

\subsubsection{Exercise 4}
I believe the inclusion diagram looks like a tree with $\Delta_5$ as the root, and the subgroups generated by
$R$ and each of the  $DR^i$ as leaves (they don't contain one another).

\subsubsection{Exercise 5}
After reflecting, it takes $2(i - 1)$ rotations to get vertex $i$ back to its original place. Thus,
reflection through vertex $i$ can be expressed as $DR^{2(i-1)}$.

\subsubsection{Exercise 6}
The two groups are the same order, so
we just need to identify two elements of $S_3 \times S_2$ with $R$ and $D$ and show that these two elements
satisfy the generator relations. Let $x = ((1 2 3), (1 2))$ and $y = ((1 3), 1)$. Then $x^6 = (1, 1)$ since
$(1 2 3)$ has order 3 and $(1 2)$ has order 2. Similarly, $y^2 = (1, 1)$ and $yx = x^{n - 1}y$, so we have
an isomorphism.

\subsubsection{Exercise 8}
(a) From $a^4 = 1$ we get that $a$ is an element of order  4. From $b^2 = a^2$, we get that only $b$, $b^3$,
$ab$, and $ab^3$, are distinct from the $a^i$. Hence, there are 8 distinct elements.

(b) There is no isomorphism to $\Delta_8$, since the only element of order 2 is $b^2 = a^2$.

\subsubsection{Exercise 10}
We let $\psi((b, c)) = bc$. This is a morphism, since $\psi((b, c)(b', c')) = uu'vv' = uvu'v'$. Additionally,
$\psi$ sends $(b, 1)$ to $u$ and $(1, c)$ to $v$. To see that $\psi$ is unique, we note that
\begin{align*}
        \psi'((b, 1)) = u, \: \psi'((1, c)) = v \implies \psi'((b, c)) = uv
\end{align*}
if $\psi'$ is a morphism. 

\subsection{Symmetric and Alternating Groups}

\subsubsection{Exercise 3}
Since $(1 2 3) (1 2) = (1 3)$ and $(1 2) (1 2 3) = (2 3)$ we have that $S_3$ is not abelian, and therefore 
$S_n$ with $n \geq 3$ is non-abelian ($S_3 \subset S_{n \geq 3}$). $S_1$ and $S_2$ are cylic and thus abelian.
As for the alternating groups, it is again straightforward to see that $A_2$ is abelian (it only consists of
the identity). $A_3$ is also abelian, since the only even permutations are $1, (1 2 3), (1 3 2)$, all of which
commute. For $n \geq 4$ though, we have that $(1 2 3) (2 3 4) \neq (2 3 4) (1 2 3)$, so $A_{n \geq 4}$ is
non-abelian.

\subsubsection{Exercise 4}
(a) The 4-element subgroup consisting of $1, (1 2), (3 4), (1 2) (3 4)$.

(b) The 6-element subgroup consisting of all of the elements in (a), plus \\ $(1 3) (2 4), (1 4) (2 3)$.

\subsubsection{Exercise 5}
There are $\binom{4}{3} = 4$ ways of choosing 3 elements in $S_4$. The subgroup generated by the 
transpositions of the 3 selected elements is isomorphic to $S_3$. 
There are $\binom{4}{2} = 6$ ways of choosing 2 elements in $S_4$. The subgroup generated by the transposition
of these two elements is ismorphic to $S_2$. Additionally, any such transposition can be paired with the
transposition of the remaining two elements (e.g. $(1 2) (3 4)$ ) to produce another subgroup isomorphic to
$S_2$, giving 9 such subgroups.

\subsubsection{Exercise 6}
The idea is the same as the second part of Exercise 5. We have $\binom{6}{3} = 20$ ways to pick 3 elements in
$S_6$, and the transpositions of these elements can then be paired with the transpositions of the remaining 3
elements to produce at least another 10 subgroups isomorphic to $S_3$ (the pairing order can be changed to
produce more).

\subsubsection{Exercise 7}
The fact that $\sigma$ and $\tau \sigma \tau^{-1}$ have the same parity follows immediately from Proposition
15 ($(x_1 \: ...\:  x_k) \to (\tau(x_1) \: ... \: \tau(x_k))$). That the two need not have the same number of inversions
can be seen from looking at $(1 2 3) (1 2) (1 3 2) = (1 3)$. The permutation $(1 2)$ only inverts $(1, 2)$,
whereas $(1 3)$ inverts $(1, 2), (1, 3), (2, 3)$.

\subsubsection{Exercise 8}
Any cycle of length $2m$ has parity $2m - 1$ (Theorem 17, Corollary 2), and is thus odd. Therefore the product
of not necessarily disjoint cycles being even implies that the product contains an even number of even length
cycles. Noting that odd cycles have even parity gives the reverse direction.

\subsubsection{Exercise 9}
A permutation of order 14 must consist of either a single cycle of order 14, two cycles of orders 2 and 7, or
a combination of both (since the order is the LCM of the disjoint cycle decomposition lengths). However, since
we are considering permutations on 10 letters, only the case of 2 and 7 is possible, which means any such
permutation must be odd.

\subsubsection{Exercise 10}
We first show that any odd length cycle can be written as a product of length 3 cycles. Let $\sigma = 
(x_1 ... x_{2n + 1})$. Then $\sigma$ can be decomposed as $(x_1 x_{2n} x_{2n+1}) \: ... \: (x_1 x_2 x_3)$, or,
in other words, the product of 3-cycles consisting of its first element $x_1$ paired with consecutive pairs
$x_{2k}, x_{2k+1}$. Next, we show that a product of two disjoint even length cycles $\sigma_1 = (x_1 \: ... \: x_{2n})$ 
and $\sigma_2 = (y_1 \: ... \: y_{2m})$ can be rewritten as the product of two odd length cycles. To do so,
we modify $\sigma_1$ to be $\sigma_1' = (x_1 \: ... \: x_{2n} y_1)$ and modify $\sigma_2$ to be 
$\sigma_2' = (y_1 \: ... \: y_{2m} x_{2n})$. We can then verify that $\sigma_1' \circ \sigma_2' = \sigma_1 \circ \sigma_2$. Thus, since any even permutation must have a disjoint cycle decomposition consisting of an even 
number of even length cycles (since they have odd parity), an even permutation can be written as the product
of 3-cycles.

\subsubsection{Exercise 11}

\subsubsection{Exercise 12}

\subsubsection{Exercise 13}
That $(1 2), (2 3), \: ...\:  (n - 1 n)$ are generators for $S_n$ follows immediately from Exercise 11 and the fact 
that $(1 2 \: ... \: n - 1) = (n - 2 n - 1) ... (1 2)$.

\subsection{Transformation Groups}

\subsubsection{Exercise 2}
The left regular representation of $S_3$ is the function that assigns each element $\sigma \in S_3$ to
$f_{\sigma} (x) = \sigma \circ x$ where $f_{\sigma}:  S_3 \to S_3$.

\subsubsection{Exercise 4}
The isotropy subgroup of a single vertex is just the group of permutations that leave the given vertex fixed
and permute the other seven vertices (isomorphic to $S_7$).

\subsubsection{Exercise 5}
The isotropy subgroups are all isomorphic to $S_{n - 1}$, as they consist of all permutations that leave a 
single element fixed. To see that these subgroups are conjugate to one another, let $G_i$ be the isotropy
subgroup of $i$. Then we have that for  $g \in G_i$, $(i j) g (i j) \in G_j$, since $(i j)$ maps $j$ to $i$ 
and then back to $j$ again.

\subsubsection{Exercise 6}
From Proposition 15, we have that cyles of the same length are conjugate to one another. Furthermore,
we know that every element of $S_n$ has a unique disjoint cycle decomposition. The different possible length
cycle decompositions form unique conjugacy classes (from Proposition 15), so the number of conjugacy classes
for $S_n$ is just the number of partitions of $n$. For $S_3$ this is 3 and for $S_4$ this is 5.

\subsubsection{Exercise 7}
The left regular representation of the additive group  $\mathbb{R}$ assigns to each element $z \in \mathbb{R}$ 
the function $f_z(x) = x + z$, which is exactly a translation by $z$ of the real line. Similarly,
the left regular representation of the additive group $\mathbb{R} \times \mathbb{R}$ corresponds to a
translation of $(z_1, z_2)$ in the cartesian plane.

\subsubsection{Exercise 8}
Since $G$ acts transitively on $x$, there exists $g$ such that $gx = y$. Let $z$ be an element that fixes 
$x$. Then we have that $gzg^{-1} y = gzx = gx = y$, so $gzg^{-1}$ fixes $y$ as desired.

\subsubsection{Exercise 9}
For any subgroup $S$ of $\Delta_4$, we can consider any action of $\Delta_4$ on the square that has each
element of an equivalence class of $\Delta_4 / S$ act the same way on an element of the square (not exactly
sure how an ``element'' of the square should be defined, I suppose a point). Such an action necessarily
fixes the subgroup $S$.

I was a bit confused by this question and found some more discussion \href{https://math.stackexchange.com/questions/76255/show-every-subgroup-of-d4-can-be-regarded-as-an-isotropy-group-for-a-suitable-ac}{here}.

\subsubsection{Exercise 10}
Let $I$ be an invariant subset containing $x$. Then $gx \in I$ for all $g \in G$, implying that 
$\text{Orb}(x) \subset I$. Since by definition  $\text{Orb}(x)$ is invariant, it must be the smallest such
set. By the previous logic, we further have that $I = \cup_{x \in G} \text{Orb}(x)$, which can be reduced
to a union of disjoint orbits (since one element appearing in another's orbit means their orbits are the same).

\subsection{Cosets}

\subsubsection{Exercise 1}
The image of a right coset $aS = \{ as \: | \: s \in S \}$ under the bijection $a \mapsto a^{-1}$ is the set 
$\{ s^{-1} a^{-1} \: | \: s \in S\}$, which is the left coset $Sa^{-1}$.

\subsubsection{Exercise 2}
The indicated subgroup $S$ is just ${D, 1}$. Thus, the left cosets consist of reflections followed by rotations
while the right cosets consist of rotations followed by reflections.

\subsubsection{Exercise 5}
Consider $x \in S / S \cap T$ and $y \in T$. By the definition of join, the product $xy$ must be in $S \vee T$.
Since $S / S \cap T$ and $T$ are disjoint, this means that $[S : S \cap T] [T : 1] \leq [S \vee T : 1]$ (since
$S \vee T$ contains every such product $xy$). Using the fact that  $[S : S \cap T] = \frac{[S : 1]}{[S \cap T : 1]}$, we then have the desired inequality.

\subsubsection{Exercise 6}
We use the result of Exercise 8 to get that a group with 6 elements is either isomorphic to $\mathbb{Z}_6$ or
to a group with 3 elements of order 2 and 2 elements of order 3. Since $S_3$ has 3 elements of order 2 and
2 elements of order 3, there is an isomorphism between the latter category of order 6 groups and $S_3$.

\subsubsection{Exercise 7}
Again we rely on Exercise 8. Since $\Delta_5$ contains 5 elements of order 2 and 4 elements of order 5,
any group of order 10 without an order 10 element is isomorphic to $\Delta_5$ by exercise 8.

\subsubsection{Exercise 8}
If a group of order $2p$ contains an element of order $2p$, then it is cyclic and thus isomorphic to
$\mathbb{Z}_{2p}$. If a group of order $2p$ does not contain an element of order $2p$, then we will show
the following:
\begin{itemize}
  \item It can have at most one subgroup of order $p$.
  \item It must have at least one subgroup or order $p$.
\end{itemize}
To see the first, we appeal to the inequality from Exercise 5. If there were two distinct subgroups of order
$p$, then their join would consist of at least $p^2 > 2p$ elements (for $p > 2$). To see the second, we 
consider the case where all  $2p - 1$ non-identity elements have order 2. Such a group must be abelian,
since we have $abab = 1 \implies ab = b^{-1} a^{-1} = ba$. Taking the join of the groups generated by $a$ 
and $b$ would then give us a subgroup of order 4, which is not possible since 4 does not divide $2p$. Thus,
there must be an element (and hence a generated subgroup) of order $p$. Combining these two results gives
that a group of order $2p$ that does not contain an element of order $2p$ must have $p-1$ elements of order
$p$ and $p$ elements of order 2.

\subsubsection{Exercise 9}
(a) Suppose $s a t = s' b t'$. Then we would have  $a = s^{-1} s' b t' t^{-1}$, so $a \in SbT$ and the 
double cosets $SaT$ and $SbT$ are equal. The alternative is that $s a t \neq s' b t'$ for all $s, s', t, t'$,
which would imply that the double cosets are disjoint. Thus, the double cosets of $G$ form a partition of
$G$, so their union is $G$.

(b) 

\subsection{Kernel and Image}

\subsubsection{Exercise 2}
If $N \triangleleft S_3$, then all elements in $N$ must have the same sign 
(since $\text{sgn}(ana^{-1}) = \text{sgn}(n)$). As there are no subgroups consisting of only odd 
permutations, we need only consider subgroups consisting of even permutations. The only such proper
subgroup of $S_3$ is $A_3$.

\subsubsection{Exercise 3}
Since $DR^i = R^{n - i}D \implies DR^i D^{-1} = R^{n - i}$, we have that $\{R^i\} \triangleleft \Delta_p$.
The only other proper subgroups of $\Delta_p$ are the order 2 groups generated by $DR^i$. However,
none of these subgroups are normal since $D DR^i D = DR^{n - i}$, which is clearly not
$DR^i$ or 1 for all $0 < i < n$. Thus, the subgroup of rotations is the only normal subgroup of $\Delta_p$.

\subsubsection{Exercise 4}
We first note that $R^j DR^i R^{-j} = DR^{i - 2j}$, and $DR^{i - 2j} = DR^i$ only when $j = 0$ or 
$j = \frac{n}{2}$. Thus, none of the subgroups generated by elements of the form $DR^i$ are normal,
and we only need to consider subgroups of the rotation subgroup. Any such subgroup
is normal, since $i$ divides $n - i$, so $\{R^i\}, \{R^{2i}\} \triangleleft \Delta_4$ and
$\{R^i\}, \{R^{2i}\}, \{R^{3i}\} \triangleleft \Delta_6$.

\subsubsection{Exercise 5}
(a) $ax = xa \implies x^{-1} a x = a$, so $Z(G)$ is normal.

(b) From Exercises 3 and 4, we can see that $DR^i \notin Z(\Delta_n)$. Furthermore, $R^i \in Z(\Delta_n)$
only if $R^i = R^{n - i}$, which is only possible if $n$ is even. Thus, $Z(\Delta_n)$ is either 1 or the
subgroup generated by $R^{\frac{n}{2}}$, with the latter being isomorphic to $\mathbb{Z}_2$.

(c) From Proposition 15, we have that $\tau \sigma \tau^{-1} = \sigma$ only if $\tau$ and $\sigma$ commute or
if $\tau = 1$. For $n > 2$, $S_n$ is not commutative, so $Z(S_n) = 1$.

\subsubsection{Exercise 6}
Let $A$ and $B$ be two normal subgroups of $G$. Then for $a \in A$, $b \in B$, and $g \in G$, we have
$g a g^{-1} g b g^{-1} = g a b g^{-1} = a' b'$ for some $a' \in A$ and $b' \in B$, so $A \vee B$ is
normal in G. Additionally, if $a \in A \cap B$, then we have that $g a g^{-1} \in A$ and
$g a g^{-1} \in B$ by normality of $A$ and $B$, so $A \cap B \triangleleft G$.

\subsubsection{Exercise 7}
(a) Let $f_{g}(x) = gxg^{-1}$. Then $f_{g'} \circ f_{g} (x) = g' g x g^{-1} g'^{-1} = f_{g' g} (x)$ so
$\text{In} (G)$ is a group under composition.

(b) Let $h \in \text{Aut}(G)$. Then $h \circ f_g \circ h^{-1} (x) = h(gh^{-1}(x)g^{-1}) = h(g) x h(g)^{-1}$,
so $\text{In} (G) \triangleleft \text{Aut} (G)$.

\subsubsection{Exercise 12}
Since $g \to ag$ is a permutation on $G$, $gT \to agT$ is a permutation on $G / T$. We also have that
$h_a \circ h_b (gT) = a (bg T) = (ab)g T = h_{ab} (gT)$ so $h: G \to S(G / T)$ is a morphism. 

\subsection{Quotient Groups}

\subsubsection{Exercise 1}
Let $\phi$ be a morphism with $\phi(R) = 0$ and $\phi(D) = 1$. Then $\phi$ is an epimorphism with kernel
$S$, so $\Delta_n / S \cong \mathbb{Z}_2$.

\subsubsection{Exercise 2}
The quotient groups of $\Delta_4$ correspond to its normal subgroups, which are:
\begin{itemize}
        \item $\{1\}, \{R\}, \{R^2\}$
        \item  $\{R^2, D\}, \{R^2, RD\}$
\end{itemize}

\subsubsection{Exercise 3}
(a) Let $\phi: 4\mathbb{Z} \to \mathbb{Z}_5$ be defined as $\phi(x) = x \mod 5$. Then $\phi$ is an epimorphism
with kernel $20\mathbb{Z}$, so $4\mathbb{Z} / 20\mathbb{Z} \cong \mathbb{Z}_5$. Similarly, we can also
define $\phi: \mathbb{Z}_6 \to \mathbb{Z}_3$ as $\phi(x) = x \mod 3$. This is another epimorphism with
kernel $3\mathbb{Z}_6$, so $\mathbb{Z}_6 / 3\mathbb{Z}_6 \cong \mathbb{Z}_3$.

(b) More generally, we can define $\phi: k \mathbb{Z} \to \mathbb{Z}_m$ as $\phi(x) = \frac{x}{k} \mod m$, 
which is an epimorphism with kernel $m k  \mathbb{Z}$. Thus, $k\mathbb{Z} / m k \mathbb{Z} \cong \mathbb{Z}_m$.
From this result, we get that 
$(\mathbb{Z} /m k \mathbb{Z}) / (k\mathbb{Z} / m k \mathbb{Z}) \cong \mathbb{Z}_{m k} / \mathbb{Z}_m$, which
is in turn isomorphic to $\mathbb{Z}_k$.

\subsubsection{Exercise 4}
I'm not really sure what counts as a ``familiar group here'', but we can consider $x \to \abs{x}$ again to
see that $Q^* / \{\pm 1 \}$ is isomorphic to the multplicative group of positive rationals.

\subsubsection{Exercise 6}
Consider $\phi: G \to \text{In} G$ where $\phi(g) = \gamma_g$ with $\gamma_g(x) = gxg^{-1}$. By construction,
$\phi$ is an epimorphism, and $\gamma_g(x) = x$ iff $g \in Z$. Thus, $G / Z \cong \text{In} G$.

\subsubsection{Exercise 7}
(a) Since $[xgx^{-1}, xhx^{-1}] = x g x^{-1} x h x^{-1} x g^{-1} x^{-1} x h^{-1} x^{-1} = x[g, h]x^{-1}$, we
have that $[G, G] \triangleleft G$. 

(b) Since $[G, G] [x, y] = [G, G] \implies [G, G] xy = [G, G] yx$, $G / [G, G]$ is abelian. Any morphism
$\phi: G \to A$ carries all of the elements of the coset $[G, G]x$ to the single element $\phi(x)$. Thus,
we can factor $\phi$ as $\phi' \circ p$, where $\phi': G / [G, G] \to A$ with $\phi'([G, G]x) = \phi(x)$.

\subsubsection{Exercise 8}
Since $G/N \cong \mathbb{Z}_5$, $G$ must be a group of order 10. This implies that  $G$ is either  $\Delta_5$
or $\mathbb{Z}_10$ (see exercise 7 in section 8). However, $\Delta_5$ does not have a normal subgroup
of order 2, so $G \cong Z_{10}$.

\subsubsection{Exercise 9}
Since $N \subset M$, we can define an epimorphism $\phi: G / N \to G / M$ that maps the coset $Ng$ to the
coset  $Mg$. The kernel of this epimorphism is $Nm$ for all $m \in M$, which is exactly $M / N$. Thus,
we get that  $(G / N) / (M / N) \cong G / M$.

