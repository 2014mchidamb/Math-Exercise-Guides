\section{Groups}

\subsection{Groups and Symmetry}

\subsubsection{Exercise 2}
Map each element $x \in \mathbb{Z}_6$ to the pair $(p_2(x), p_3(x))$.
This is an isomorphism, since the projections $\mathbb{Z}_6 \to \mathbb{Z}_3$ 
and $\mathbb{Z}_6 \to \mathbb{Z}_3$ are both group morphisms, and the mapping
itself is a bijection.

\subsubsection{Exercise 3}
To see that there is no isomorphism $f: \mathbb{Z}_4 \to \mathbb{Z}_2 \times \mathbb{Z}_2$, 
consider $f(1)$ and $f(3)$. We have that $f(0) = f(1 + 3) = f(1) + f(3)$ which is not
possible since $f(0) = (0, 0)$ (has to be the case since $f(x) = f(0) + f(x)$ ).

Rotations do not preserve symmetry for rectangles, since distances between adjacent
vertices change. The only transformations that preserve symmetry are reflections
across the vertical and horizontal axes, giving 4 possible transformations. We can
then map $(0, 0)$ to the identity, $(0, 1)$ to a vertical reflection, $(1, 0)$ to
a horizontal reflection, and $(1, 1)$ to a vertical + horizontal reflection.

\subsubsection{Exercise 4}

\subsubsection{Exercise 5}

\subsubsection{Exercise 6}

\subsubsection{Exercise 10}
The set of these permutations has identity  $(1, 0)$, and any permutation $(a, b)$ has
inverse $(\frac{1}{a}, -\frac{b}{a})$. Furthermore, $(a_2, b_2) \circ (a_1, b_1) = (a_1 a_2, a_2 b_1 + b_2)$,
which is associative since multiplication and addition are both associative.

\subsubsection{Exercise 11}
(a) To show that the given function is a permutation on $\mathbb{R}\cup{\infty}$, we need to show that it is a
bijection from $\mathbb{R}\cup{\infty} \to \mathbb{R}\cup{\infty}$.
Suppose $f(x_1) = f(x_2)$. Then
\begin{align*}
        \frac{ax_1 + b}{cx_1 + d} &= \frac{ax_2 + b}{cx_2 + d} \\
        (ad - bc)x_1 &= (ad - bc)x_2 \implies x_1 = x_2
\end{align*}
So $f$ is an injection from $\mathbb{R}\cup{\infty} \to \mathbb{R}\cup{\infty}$. Furthermore, if
we set $f(x) = y$, we can solve for $x$, which gives us that $f$ is also a surjection.

(b) I'm sure an inverse can be found, but it's tedious... Associativity then follows again from associativity
of multiplication and addition.

\subsubsection{Exercise 12}

\subsubsection{Exercise 13}
(a) Any automorphism of $\mathbb{Z}_3$ has to fix 0. Thus, the only two automorphisms are the identity
and the automorphism that swaps 1 and 2.

(b) Fixing $(0, 0)$, we see that we can permute the remaining three elements as we want, giving
the isomorphism to $S_3$.

(c) 

\subsection{Rules of Calculation}

\subsubsection{Exercise 1}
(a) Multiply by inverse and use associativity.

(b) Associativity.

(c) Associativity and then inverse of product.

\subsubsection{Exercise 2}
Multiply by $a^{-1}$.

\subsubsection{Exercise 3}
Since the unit is its own inverse, we're left with $2n - 1$ elements that need to be paired with one another.
Since $2n - 1$ is odd, we have that one of the elements must be its own inverse.

\subsubsection{Exercise 4}
Any group with 3 elements must be of the form $1, a, a^{-1}$. Thus, each of these groups is clearly
isomorphic to the others.

\subsubsection{Exercise 5}
I struggled to untie the ideas of cancellation and inverse, so I ended up looking up a hint for this one. To
see that an infinite set with cancellation does not need to be a group, consider $(\mathbb{N}, +)$.
This is a monoid that was proven to have cancellation in chapter 1, but does not contain inverses.

For the case of a finite set $G$, we can use the fact that $f(x) = ax$ is an injection for any $a \in G$, 
since $ax = ay \implies x = y$ by cancellation. Since $G$ is finite, $f$ is also a surjection. Therefore,
$\exists a \: | \: ax = 1$ which gives us that there is a left inverse. Applying the same logic using
$f(x) = xa$ gives a right inverse, which completes the proof since these inverses must be equal.

\subsubsection{Exercise 6}
Left cancellation is possible due to left inverse and left unit.
Furthermore, $uu = u \implies (a' a) u = a' a \implies a u = a$ by left cancellation, indicating that 
$u$ is also a right unit. Then we have that $u a' = a' u \implies a' a a' = a' u \implies a a' = u$, 
and $a'$ is also a right inverse. This proves that $X$ is a group.

\subsubsection{Exercise 7}
We proceed as directed in the hint. Since the equation $ua = a$ has solution $u$, and any $b$ 
can be written as $b = ay$, we have $u b = u (ay) = ay = b$. Thus, $u$ is a left unit. Since
the equation $a'a = u$ also has a solution $a'$, we are done by Exercise 6.

\subsubsection{Exercise 10} 
Since each element of $G$ has a unique inverse, $f(a) = a^{-1}$ is a bijection. Additionally,
$f(ab) = (ab)^{-1} = b^{-1} a^{-1} = f(b) f(a) = f(a) \square^{\text{op}} f(b)$.

\subsubsection{Exercise 11}
Associativity of $\square$ immediately follows from the associativity of $G$ 's binary operation and
the fact that $p$ is a morphism. Additionally, since $p$ is an epimorphism, $\forall x, \: \exists g
\: | \: x = p(g)$. Since $u g = g u$, $p(u)$ is then the unit for $X$. Similarly, $p(g')$ is the
inverse of $x$, thus making $X$ a group.

\subsubsection{Exercise 12}
$b b_R = u \implies b_L b b_R = b_L \implies b_R b = b_L b = u$.

\subsection{Cyclic Groups}
We first show that $\mathbb{Z}_n$ is generated only by those $c$ that are coprime to $n$. If 
$c$ is coprime to $n$, then $ac = 0$ only when $a = n$ since $c$ and  $n$ share no prime factors.
Thus, the subgroup generated by $c$ has order $n$ and is therefore all of $\mathbb{Z}_n$. Similarly, 
if $c$ is a generator of $\mathbb{Z}_n$, then $c$ has order $n$ and must therefore be coprime to $n$.

\subsubsection{Exercise 1}
The only possible generators are 1 and 5, since those are the only elements of $\mathbb{Z}_6$ 
that are coprime to 6.

\subsubsection{Exercise 2}
The endomorphisms of $\mathbb{Z}_n$ are completely determined by the mapping of 1, so there are
only $n$ such endomorphisms.

\subsubsection{Exercise 3}
5 is prime, so all elements of $\mathbb{Z}_5$ other than 0 are coprime to it. 

\subsubsection{Exercise 4}
14 has 6 positive integers less than it that are coprime to it (3, 5, 7, 9, 11, 13).

\subsubsection{Exercise 5}
The two generators of $\mathbb{Z}$ are $1$ and $-1$, as elements of $\mathbb{Z}$ can be written as $-m$ or $m$.

\subsubsection{Exercise 7}
If $G$ is abelian, then $(g_1 g_2)^m$ can be rearranged to $g_1^m g_2^m$. If $(g_1 g_2)^m = g_1^m g_2^m$.
The reverse direction follows from the $m = 2$ case, $g_1 g_2 g_1 g_2 = g_1^2 g_2^2$.

\subsubsection{Exercise 8}
$(g_1 g_2) (g_1 g_2) = 1 \implies g_1 g_2 = g_2 g_1$.

\subsubsection{Exercise 9}
The automorphisms are all determined by the mappings of the generators; the isomorphisms follow from the
number of generators of each group.

\subsubsection{Exercise 10}

\subsection{Subgroups}

\subsubsection{Exercise 1}
The subgroup mapping a given diagonal to itself consists of $\{1, R^3, D, D'\}$, where
$R^3$ is 3 clockwise rotations, $D$ is reflection across the given diagonal, and $D'$ is
reflection across the diagonal perpendicular to the given. Mapping those elements to
$\{(0, 0), (1, 1), (0, 1), (1, 0)\}$ (in order) is an isomorphism.

\subsubsection{Exercise 4}
If $S$ is closed under product and inverse, then it contains the identity and is thus a subgroup.

\subsubsection{Exercise 5}
We have that $(s, t) (t, s) = st^{-1} ts^{-1} $, so $S$ contains the identity. Then $(1, s) = s^{-1}$ and
$(s, t^{-1}) = st$, so $S$ is closed under products and inverses as well, thus making it a subgroup.

\subsubsection{Exercise 6}
(a) The identity has order 1. Additionally, if $a$ has finite order, so does $a^{-1}$. Finally,
$a^n = 1, \: b^k = 1 \implies (ab)^{nk} = a^{nk} b^{nk} = 1$.

(b) If non-abelian, we do not necessarily have $(ab)^{nk} = a^{nk} b^{nk}$.

\subsubsection{Exercise 7}
If $G$ has no proper subgroups, then it is generated by all of its non-identity elements. 
This is only possible if $G$ has order 1 (vacuously true), or if $G$ is a cyclic group of
prime order (as was shown in the beginning of the previous section).

\subsubsection{Exercise 8}
(a) If $a$ has order $n$, so does $a^{-1}$. Additionally, $(ab)^n = a^n b^n = 1$, making all elements that
satisfy $a^n = 1$ a subgroup of $A$. To see that this is not true for non-abelian groups, consider $S_3$.
The elements $(1 2)$ and $(2 3)$ are both of order 2, but $(1 2) (2 3) = (1 2 3)$ is of order 3.

(b) That the $n^{\text{th}}$ powers form a subgroup follows from $a^n a^{-n} = 1$ and $a^n b^n = (ab)^{n}$.

\subsubsection{Exercise 9}
If $T$ is a submonoid of $S$, then $i: T \to S$ is a morphism of monoids, so $T$ must necessarily be
closed under products and identity. For the reverse direction, if $T$ is closed under products and
identity, then the insertion $i$ is a morphism of monoids and $T$ is a submonoid of $S$.

\subsection{Defining Relations}

\subsubsection{Exercise 3}
The subgroup of rotations is isomorphic to $\mathbb{Z}_5$, so each element other than the identity has order 5.
The element $D$ has order 2. Furthermore, we have from the generator relations that
\begin{align*}
        DR = R^{n-1}D \implies DR^i = R^{n-1} DR^{i-1} = DR^i = R^{i(n - 1)}D = R^{n - i}D
\end{align*}
So all elements of the form $DR^i$ also have order 2.

\subsubsection{Exercise 4}
I believe the inclusion diagram looks like a tree with $\Delta_5$ as the root, and the subgroups generated by
$R$ and each of the  $DR^i$ as leaves (they don't contain one another).

\subsubsection{Exercise 5}
After reflecting, it takes $2(i - 1)$ rotations to get vertex $i$ back to its original place. Thus,
reflection through vertex $i$ can be expressed as $DR^{2(i-1)}$.

\subsubsection{Exercise 6}
The two groups are the same order, so
we just need to identify two elements of $S_3 \times S_2$ with $R$ and $D$ and show that these two elements
satisfy the generator relations. Let $x = ((1 2 3), (1 2))$ and $y = ((1 3), 1)$. Then $x^6 = (1, 1)$ since
$(1 2 3)$ has order 3 and $(1 2)$ has order 2. Similarly, $y^2 = (1, 1)$ and $yx = x^{n - 1}y$, so we have
an isomorphism.

\subsubsection{Exercise 8}
(a) From $a^4 = 1$ we get that $a$ is an element of order  4. From $b^2 = a^2$, we get that only $b$, $b^3$,
$ab$, and $ab^3$, are distinct from the $a^i$. Hence, there are 8 distinct elements.

(b) There is no isomorphism to $\Delta_8$, since the only element of order 2 is $b^2 = a^2$.

\subsubsection{Exercise 10}
We let $\psi((b, c)) = bc$. This is a morphism, since $\psi((b, c)(b', c')) = uu'vv' = uvu'v'$. Additionally,
$\psi$ sends $(b, 1)$ to $u$ and $(1, c)$ to $v$. To see that $\psi$ is unique, we note that
\begin{align*}
        \psi'((b, 1)) = u, \: \psi'((1, c)) = v \implies \psi'((b, c)) = uv
\end{align*}
if $\psi'$ is a morphism. 
