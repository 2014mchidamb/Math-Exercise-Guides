\section{Spectral Theory}

\subsection{Exercise 1}
(a) We can re-express $h$ as a linear combination of its eigenvectors,
from which we can see that $A^n$ causes all of these components to go to 0.

(b) Same as in part (a), except all of the components now go to $\infty$.

\subsection{Exercise 2}
$A (A^N f) = A(a^N f + Na^{N - 1} h) = a^{N+1} f + a^N h + Na^N h$.

\subsection{Exercise 3}
Suppose $q(A) = \sum_{i = 0}^N q_i A^i$. Then $q_i A^i f = q_i a^i f + q_i i a^{i-1} h$ by
Exercise 2. From the linearity of the derivative it then follows that $q(A) f = q(a) f + q'(a) h$. 

\subsection{Exercise 4}
Applying Lemma 9 to $p_1 ... p_k$ and $p_{k+1}$ gives the desired result.

\subsection{Exercise 5}
$(A - aI)^d x = 0 \implies A (A - aI)^d x = 0 \implies (A - aI)^d (Ax) = 0 \implies Ax \in N_d$.

\subsection{Exercise 6}
Since $m_A$ divides the characteristic polynomial of $A$ (by definition of $d_i$), $m_A (A) = 0$ by Cayley-Hamilton. 
Suppose there is some polynomial $q(A) = 0$ with $\deg(q) < \deg(m_A)$.
The roots of $q$ must contain all of the eigenvalues of $A$, since for any eigenvector $h$ we have that
$q(A)h = q(a_h)h$ (where $a_h$ denotes the eigenvalue associated with $h$). Thus, the roots of $q$ can
only differ in multiplicity from $m_A$. However, if any root of $q$ has multiplicity $d_i' < d_i$, then
we can choose an element $x \in N_{d_i}$ such that $q(A)x \neq 0$, which is a contradiction (since
$N_{d_i'} \subset N_{d_i}$).

\subsection{Exercise 7}
The columns of $A$ are $Ax_i$.

\subsection{Exercise 8}
By induction.

\subsection{Exercise 9}
The minimal polynomial of $A$ divides the minimal polynomial of $A^{\top}$, and vice versa,
so they must be the same. Thus, the indices of each eigenvalue of $A$ and $A^{\top}$ must be the same,
which means we can apply Theorem 12 to see that they are similar.

\subsection{Exercise 10}
$(\xi^{(i)}, x) = \sum k_j (\xi^{(i)}, x^{(j)}) = k_i (\xi^{(i)}, x^{(i)})$ since
$(\xi^{(i)}, x^{(j)}) = 0$ by Theorem 17.
