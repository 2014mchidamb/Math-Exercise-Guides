\section{Calculus of Vector and Matrix Valued Functions}

\subsection{Exercise 1}
That $\dot{x}(t) = 0 \implies x(t) = c$ follows immediately from the mean value inequality for vector-valued
functions, $\norm{x(b) - x(a)} \leq (b - a) \norm{x'(t)}$ for some $t \in (a, b)$. My guess is that
Lax was hinting at applying the mean value theorem for real-valued functions to $(x(b) - x(a), y)$.

\subsection{Exercise 2}
We have that
\begin{align*}
        \dv{x} A^{-1} A = 0 = \bigg(\dv{x} A^{-1}\bigg) A + A^{-1} \bigg(\dv{x} A\bigg) \implies \dv{x} A^{-1} = -A^{-1} \bigg( \dv{x} A \bigg) A^{-1} 
\end{align*}

\subsection{Exercise 3}
The matrix $A + B$ satisfies $(A + B)^2 = I$, so we have that
\begin{align*}
        e^{A+B} &= \sum_{k = 0}^\infty \frac{(A + B)^k}{k!} \\
                &= \bigg(\sum_{k = 0}^\infty \frac{1}{(2k)!}\bigg) I +  \bigg(\sum_{k = 0}^\infty \frac{1}{(2k + 1)!}\bigg) (A + B) \\
                &= \cosh(1) I + \sinh(1) (A + B)
\end{align*}

\subsection{Exercise 4}
We can use uniform continuity to swap the limits to get
\begin{align*}
        \lim_{m \to \infty} \norm{\dot{E}_m(t) - F(t)} &= \lim_{m \to \infty} \lim_{h \to 0} \norm{\frac{E_m(t + h) - E_m(t)}{h} - F(t)} \\
                                                       &= \lim_{h \to 0} \lim_{m \to \infty} \norm{\frac{E_m(t + h) - E_m(t)}{h} - F(t)} \\
                                                       &= \lim_{h \to 0} \norm{\frac{E(t + h) - E(t)}{h} - F(t)} \\
                                                       &\implies \dot{E}(t) = F(t)
\end{align*}

\subsection{Exercise 5}
Let $M \geq \norm{A(t)}, \norm{\dot{A}(t)}$. Then we can use the expression for the derivative of $A^k(t)$
to get
\begin{align*}
        \norm{\dv{t} A^k} &= \norm{\dot{A} A^{k - 1} + A \dot{A} A^{k - 2} + ... + A^{k - 1}\dot{A}} \\
                   &\leq kM^k \\
        \implies \norm{\dot{E}_m(t) - \dot{E}_n(t)} &\leq \sum_{k = n + 1}^m \frac{kM^k}{k!}
\end{align*}
so $\dot{E}_m(t)$ converges.

\subsection{Exercise 6}
\begin{align*}
        \dv{t} \log\det e^{At} = \Tr(e^{-At} A e^{At}) = \Tr(A)
\end{align*}

\subsection{Exercise 7}
Let $v$ be an eigenvector of $A$ with eigenvalue $a$. Then
\begin{align*}
        e^A v &= \sum_{k = 0}^\infty \frac{A^k v}{k!} \\
              &= \sum_{k = 0}^\infty \frac{a^k v}{k!} \\
              &= e^a v
\end{align*}

