\section{Determinant and Trace}

\subsection{Exercise 1}
(a) The discriminant already has ordered versions of all the $(i, j)$ difference terms.
Applying a permutation only changes the signs of some of the difference terms, hence
$\sigma(p) = 1, -1$.

(b) $\sigma(p_1 \circ p_2) = \text{sign}(P(p_1 \circ p_2 (x_1, ..., x_n)))  = \sigma(p_1) \text{sign}(P(p_2 (x_1, ..., x_n)))$.

\subsection{Exercise 2}
(c) A transposition swaps two indices, and hence flips the sign of their associated
difference term in the discriminant.

(d) If $p(i) = j$, then we can start with the permutation $(i \: j)$.
Next, if $p(j) = k$, we can compose with  $(i \: k)$ to get $(i \: k) \circ (i \: j)$.
We can do this until we have completely reconstructed the permutation using
transpositions.

\subsection{Exercise 3}
By starting with a different $i$ in Exercise 2 (d), we can obtain a different
decomposition of transpositions. However, the parity of the decomposition must
be the same, as otherwise $\sigma(p)$ will take on two different values for
the same $p$.

\subsection{Exercise 4}
(Property II): Each term in $D(a_1, ..., a_n)$ contains exactly one element
from each of the $a_i$. Thus, scaling any of the $a_i$ by $k$ scales the entire
determinant by $k$. Similar logic for vector addition.

(Property III): The only non-zero term in $D(e_1, ..., e_n)$ is associated with
the identity permutation, hence $D(e_1, ..., e_n) = 1$.

(Property IV): Swapping two arguments is the same as applying a transposition to
each of the terms in $D(a_1, ..., a_n)$, which flips the sign of $D$.

\subsection{Exercise 5}
Suppose $a_1 = a_2$. Then:
\begin{align*}
        D(a_1, a_2, ..., a_n) &= -D(a_2, a_1, ..., a_n) \\
        D(a_1, a_2, ..., a_n) + D(a_1, a_2, ..., a_n) &= 0
\end{align*}

\subsection{Exercise 6}
We can swap rows and columns until $A$ is in the same form as in Lemma 2.
Since each row and column swap is equivalent to applying a transposition,
we get that $\det A = (-1)^{i + j} \det A_{ij}$.
