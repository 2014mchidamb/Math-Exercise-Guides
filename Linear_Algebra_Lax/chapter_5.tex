\section{Determinant and Trace}

\subsection{Exercise 1}
(a) The discriminant already has ordered versions of all the $(i, j)$ difference terms.
Applying a permutation only changes the signs of some of the difference terms, hence
$\sigma(p) = 1, -1$.

(b) $\sigma(p_1 \circ p_2) = \text{sign}(P(p_1 \circ p_2 (x_1, ..., x_n)))  = \sigma(p_1) \text{sign}(P(p_2 (x_1, ..., x_n)))$.

\subsection{Exercise 2}
(c) A transposition swaps two indices, and hence flips the sign of their associated
difference term in the discriminant.

(d) If $p(i) = j$, then we can start with the permutation $(i \: j)$.
Next, if $p(j) = k$, we can compose with  $(i \: k)$ to get $(i \: k) \circ (i \: j)$.
We can do this until we have completely reconstructed the permutation using
transpositions.
