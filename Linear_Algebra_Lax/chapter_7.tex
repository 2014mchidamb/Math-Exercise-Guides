\section{Euclidean Structure}

\subsection{Exercise 1}
Applying Cauchy-Schwarz gives that $(x, y) \leq \norm{x}\norm{y} = \norm{x}$, which yields
the desired result.

\subsection{Exercise 2}
Both of these just follow from the fact that a linear space with Euclidean structure is
isomorphic to $R^k$, which has the desired properties. The following feel like cop-out
solutions, but I feel they're fair given this isn't supposed to be a real analysis text.

(i) Let $x_k = \sum_i a_i^{(k)} x^{(i)}$ and $x_j = \sum_i a_i^{(j)} x^{(i)}$. Then we have that
$\norm{x_k - x_j} \to 0 \implies \abs{a_i^{(k)} - a_i^{(j)}} \to 0$. Since $\mathbb{R}$ is
complete, $a_i^{(n)} \to a_i$ for some $a_i \in \mathbb{R}$. Thus $x_k \to x = \sum a_i x^{(i)}$.

(ii) Same logic as (i): the individual $a_i^{(n)}$ have convergent subsequences.

\subsection{Exercise 3}
We also need to assume $X$ is finite dimensional (I think).

(i) Since $X$ is finite dimensional, we have that $x = \sum a_i e^{(i)}$ for some basis $e^{(i)}$. Let
$v$ be the vector whose components are $v_i = \norm{Ae^{(i)}}$. Then
\begin{align*}
        \norm{Ax} &= \norm{\sum a_i Ae^{(i)}} \\
                  &= \norm{(x, v)} \\
                  &\leq \norm{x} \norm{v}
\end{align*}
Since $\norm{v}$ is a constant, $\norm{Ax}$ is bounded on the unit sphere.

(ii) We have that $\norm{A} = \max_x \frac{\norm{Ax}}{\norm{x}} = \max_x \norm{Ax}$.
 The result then follows from $(Ax, y) \leq \norm{Ax}$.

(iii) Let $v$ be as in (i). Then $\norm{Ax_i - Ax_j} \leq \norm{A} \norm{x_i - x_j}$. Since
$\norm{A} \leq \norm{v}$ from (i), we are done (we can make $ \norm{x_i - x_j}$ as small
as we'd like).

\subsection{Exercise 4}
Follows immediately from $(Ax, y) = (x, A^{*}y)$ and Exercise 3 (ii).

\subsection{Exercise 5}
Let $x = y_1 + y_1^{\perp}$ and $z = y_2 + y_2^{\perp}$. Then
\begin{align*}
        (P_Y x, z) &= (y_1, y_2 + y_2^{\perp}) \\
                   &= (y_1, y_2) + 0 \\
                   &= (y_1, y_2) + (y_1^{\perp}, y_2) \\
                   &= (x, P_Y z)
\end{align*}

\subsection{Exercise 6}
Reflection across $x_3 = 0$ sends $(x_1, x_2, x_3) \to (x_1, x_2, -x_3)$. Hence it can be represented as
\begin{align*}
       A =
       \begin{pmatrix}
               1 & 0 & 0 \\
               0 & 1 & 0 \\
               0 & 0 & -1
       \end{pmatrix}
\end{align*}
Which has determinant -1 (product of diagonal terms).

\subsection{Exercise 7}
(a) If $A$ is orthogonal, then $A^{*}A = I$. Since $A^{*} = A^{\top}$ in this case, we immediately get
that the column vectors of $A$ are pairwise orthogonal unit vectors. Similarly, a matrix $A$ with
pairwise orthogonal unit vectors satisfies $A^{*} A = I$, implying that it is orthogonal.

(b) $A^{*} A = I \implies A^{*} = A^{-1} \implies A A^{*} = I$ so $A$ orthogonal implies $A^{*}$ orthogonal.
The result then follows from plugging $A^{*}$ into (a).

\subsection{Exercise 8}
The proof is almost identical to the non-complex case.
\begin{align*}
        (x + ty, x + ty) &= (x, x) + (ty, x) + (x, ty) + \norm{t}^2 (y, y) \\
                         &= (x, x) + t(y, x) + \bar{t}(x, y) + \norm{t}^2 (y, y)
\end{align*}
Plugging in $t = \frac{(x, y)}{(y, y)}$ and using the fact that $(x + ty, x + ty) \geq 0$ for all
complex $t$ gives the desired inequality.

\subsection{Exercise 9}
\subsubsection{Theorem 4}
The proof of Theorem 4 is the same, except we use $y = \sum \bar{b_k} x^{(k)}$ instead.

\subsubsection{Theorem 5}
No changes need to be made to the proof of Theorem 5.

\subsubsection{Theorem 6}
Again, no changes need to be made.

\subsubsection{Theorem 7}
No changes need to be made here either, since $(y, y^{\perp}) = 0 \implies \overline{(y, y^{\perp})} = 0$.

\subsection{Exercise 10}
\subsubsection{Theorem 8}
Parts (i)-(iii) remain the same. For part (iv), we see that 
\begin{align*}
        (Ax, y) = (x, A^{*}y) = \overline{(A^{*}y, x)} = \overline{(y, A^{**}x)} = (A^{**}x, y)
\end{align*}

\subsubsection{Theorem 9}
(i) $\norm{kA} = \max_x \frac{\sqrt{(kAx, kAx)}}{\norm{x}} = \sqrt{k\bar{k}} \norm{A} = \norm{k} \norm{A}$.

(ii) Applying triangle inequality to the definition of norm gives the result.

(iii) Comes immediately from $\norm{A(Bx)} \leq \norm{A} \norm{Bx}$.

\subsection{Exercise 11}
Follows from $\norm{M(x) - M(y)} = (\bar{x} - \bar{y}) (x - y) = \norm{x - y}$.

\subsection{Exercise 12}
Same idea as proof of Theorem 10.

\subsection{Exercise 13}
As in Exercise 7, $M^{*} M = I \implies M^{*} = M^{-1} \implies M M^{*} = I$. Similarly,
$M^{*} = M^{-1} \implies (M^{*})^{-1} M^{-1} = I$.

\subsection{Exercise 14}
Associativity follows from associativity of composition. From Exercise 13, if $M$ is unitary, so is $M^{-1}$.
Finally, $I$ is also unitary, so the unitary maps are a group with unit $I$.

\subsection{Exercise 15}
Again, same idea as in Theorem 10: $\det M^{*} \det M = 1$.

\subsection{Exercise 16}
\begin{align*}
        (Mf, Mg) = \int_{-1}^{1} m^2(s) f(s) \bar{g}(s) = \int_{-1}^{1} f(s) \bar{g}(s)
\end{align*}

