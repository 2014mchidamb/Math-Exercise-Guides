\section{Transformations and Expectations}

\subsection{Exercise 1}
(a) $F_Y(y) = P(X^3 \leq y) = F_X (y^{\frac{1}{3}})$. Differentiating, we get that 
$f_Y(y) = 14y(1 - y^{\frac{1}{3}})$ which integrates to 1 over $(0, 1)$.

(b)  $f_Y(y) = \dv{y} F_X(\frac{y - 3}{4}) = \frac{7}{4} \exp(\frac{-7(y - 3)}{4})$, which again
integrates to 1 over $(3, \infty)$. The sample space of $Y$ consists of $(3, \infty)$ because 
$4X + 3$ is monotonically increasing and $4(0) + 3 = 3$.

(c) We need only consider positive square roots, since  $x \in (0, 1)$. Thus,
$f_Y(y) = \frac{1}{2\sqrt{y}}30y (1 - \sqrt{y})^2$, which integrates to 1 over $(0, 1)$.

\subsection{Exercise 2}
(a) $f_Y(y) = \frac{1}{2\sqrt{y}} f_X(\sqrt{y}) = \frac{1}{2\sqrt{y}}$

(b) Plug $x = e^{-y}$ into $f_X(x)$ and multiply by $e^{-y}$ (the negative of the derivative, since $-\log X$
is decreasing).

(c) Plug in  $x = \log y$ and multiply by $\frac{1}{y}$.

\subsection{Exercise 3}
From the definition of $Y$, we can see that the sample space is
$\mathcal{Y} = \big(\frac{n}{n+1}\big)_{\mathbb{N}}$. Solving $y = \frac{x}{x + 1}$ for $y$, we find that
$f_Y(y) = f_X(\frac{y}{1 - y})$.

\subsection{Exercise 4}
(a) Integrating from $-\infty$ to 0 gives $\frac{1}{2}$ and likewise for 0 to  $\infty$, so $f$ is a pdf.

(b) If $t \leq 0$, then we have that $P(X < t) = \frac{1}{2} e^{\lambda t}$. Otherwise, 
\begin{align*}
        P(X < t) &= \int_{-\infty}^{t} f(x) = \int_{-\infty}^{0} f(x) + \int_{0}^{t} f(x) \\ 
                 &= \frac{1}{2} + \frac{1}{2} - \frac{1}{2} e^{-\lambda t} \\
                 &= 1 - \frac{1}{2}e^{-\lambda t}
\end{align*}

(c) If $t \leq 0$, $P(\abs{X} < t)$ is clearly 0. Otherwise,
\begin{align*}
        P(\abs{X} < t) &= P(-t < X < t) = P(X < t) - P(X < -t) \\
                       &= 1 - e^{- \lambda t} 
\end{align*}
from the CDFs computed in part (b). 

\subsection{Exercise 5}
The sample space $\mathcal{Y}$ is $[0, 1]$. Thus, we need to consider 
$P(Y \leq y) = P(X \leq \sin^{-1} \sqrt{y})$ for $y \in [0, 1]$. By symmetry, we can just consider the case
where $\sin^{-1}$ is restricted to $[0, \frac{\pi}{2}]$, as the other three quadrants have the same area.
Therefore,
\begin{align*}
        f_Y(y) &= 4 \dv{y} F_X(\sin^{-1} \sqrt{y}) \\
               &= 4 \dv{y} \frac{\sin^{-1} \sqrt{y}}{2\pi} \\
               &= \frac{1}{\pi \sqrt{y (1 - y)}}
\end{align*}
At $y = 0$ and $y = 1$, the density is infinite/undefined - I'm not really sure how to interpret this.

\subsection{Exercise 11}
(a) We have that 
\begin{align*}
        \mathrm{E} [X^2] = \frac{1}{\sqrt{2\pi}}\int_{-\infty}^{\infty} x^2 e^{-\frac{x^2}{2}} &= \frac{2}{\sqrt{2\pi}} \int_{0}^{\infty} x^2 e^{-\frac{x^2}{2}} \\
                                                                            &= \frac{2}{\sqrt{2\pi}} \big(\lim_{x \to \infty} -x e^{-\frac{x^2}{2}} + \int_{0}^{\infty} e^{-\frac{x^2}{2}}\big) \\
                                                                            &= 1
\end{align*}
Where we integrated by parts using $dv = xe^{-\frac{x^2}{2}}$ and $u = x$, applied L'H\^opital's rule, 
and then compared to the CDF of the normal distribution.

(b) The pdf $f_Y(y)$ is just $f_X(y) + f_X(-y)$, so $f_Y(y) = \frac{2}{\sqrt{2\pi}} e^{-\frac{y^2}{2}}$.
We can then compute $\mathrm{E} [Y]$ as
\begin{align*}
        \mathrm{E} [Y] = \frac{2}{\sqrt{2\pi}} \int_{0}^{\infty} y e^{-\frac{y^2}{2}} &= \frac{2}{\sqrt{2\pi}}  
\end{align*}
To compute the variance we need to compute $\mathrm{E} [Y^2]$, which is identical to $\mathrm{E} [X^2]$. Thus, 
$\mathrm{Var} [Y] = \mathrm{E} [Y^2] - \mathrm{E} [Y]^2 = 1 - \frac{2}{\pi}$.
