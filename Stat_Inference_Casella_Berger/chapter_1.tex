\section{Probability Theory}

\subsection{Exercise 1}
(a) Four character strings consisting of H and T (16 total).

(b) $\mathbb{N}$

(c) $\mathbb{R}^+$

(d) $\mathbb{R}^+ / 0$

(e) $\frac{i}{n}$ for $i = 0, ..., n$.

\subsection{Exercise 4}
(a) $P(A) + P(B) - P(A \cap B)$ 

(b) $P(A) + P(B) - 2P(A \cap B)$ 

(c) Same as (a).

(d) $1 - P(A \cap B)$

\subsection{Exercise 6}
We have $u + w = 1 \implies u^2 + 2uw + w^2 = 1$. However, it is not possible for
$u^2, 2uw, w^2$ to all equal $\frac{1}{3}$; thus, there are no such $u, w$.

\subsection{Exercise 12}
(a) Just a case of countable additivity.

(b) Split $\cup_{i = 1}^{\infty} A_i$ into $\cup_{i = 1}^{n} A_i$ and $\cup_{i = n + 1}^{\infty} A_i$.
Taking $n \to  \infty$ sends the probability of the latter union to 0, leaving the
desired result.

\subsection{Exercise 13}
No, since $P(B^c) = 1 - P(B) \implies P(B) = \frac{3}{4}$ and $\frac{3}{4} + \frac{1}{3} > 1$.

\subsection{Exercise 18}
This exercise (based on the provided answer) seems to assume that the balls are distinguishable. 
There are $n^n$ different arrangements of balls in cells (each ball has $n$ options). If 
exactly one cell must remain empty, then exactly one other cell must have two balls. There are
$n$ choices for the empty cell and $n - 1$ choices for the double cell. The two balls to go into
the double cell can be chosen in $\binom{n}{2}$ ways, and the remaining balls can be ordered in
$(n - 2)!$ ways (since they are distinguishable), hence giving the provided result.

\subsection{Exercise 21}
There are $\binom{2n}{2r}$ ways to choose $2r$ shoes from the collection. If we want to ensure that
no pairs of shoes are drawn, we should only draw one shoe from each of the $n$ pairs, which we can do
in $\binom{n}{2r}$ ways. For each of the $2r$ shoes we can either choose the right or left shoe from 
the pair (assuming distinguishability).

\subsection{Exercise 23}
We use \href{https://en.wikipedia.org/wiki/Vandermonde%27s_identity}{Vandermonde's identity} with $m = n = r$ to get
\begin{align*}
        \sum_{k = 0}^n \bigg(\frac{1}{2^n} \binom{n}{k} \bigg)^2 = \frac{1}{4^n} \binom{2n}{n}
\end{align*}

\subsection{Exercise 25}
Depends on how you define the sample space, see \href{https://en.wikipedia.org/wiki/Boy_or_Girl_paradox#Analysis_of_the_ambiguity}{Wikipedia} for a lengthy discussion.

\subsection{Exercise 26}
We take the complement of the probability that we roll a 6 in the first 5 rolls:
$1 - \sum_{k = 0}^{4} \frac{5^k}{6^{k+1}}$.

\subsection{Exercise 35}
From the definition of conditional probability and the fact that $P(B) > 0$,
we have that $P(\dot | B) > 0$ and $P(S | B) = \frac{P(B)}{P(B)} = 1$. Countable additivity
follows from $A \cap B$ and $A' \cap B$ being disjoint if $A$ and $A'$ are.

\subsection{Exercise 36}
To solve the first part, we take the complement of the probability that the target was hit fewer than
two times, which is
\begin{align*}
        1 - \bigg(\frac{4}{5}\bigg)^{10} - 10\bigg(\frac{4}{5}\bigg)^9 \frac{1}{5} \approx 0.624
\end{align*}
If we let $A$ be the event that the target was hit at least once, and $B$ be the event that the target was
hit at least twice, then we see that $A \cap B = B$. The conditional probability is thus
\begin{align*}
        \frac{1 - \bigg(\frac{4}{5}\bigg)^{10} - 10\bigg(\frac{4}{5}\bigg)^9 \frac{1}{5}}{1 - \bigg(\frac{4}{5}\bigg)^{10}} \approx 0.699
\end{align*}
