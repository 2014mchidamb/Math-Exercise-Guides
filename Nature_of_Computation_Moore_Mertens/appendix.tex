\section{Appendix}

\subsection{The Story of O}

\subsubsection{Exercise A.1}
We can choose $n_3 = max(n_1, n_2)$ such that $f_1(n) + f_2(n) \leq (C_1 + C_2)g$ whenever
$n > n_3$.

\subsubsection{Exercise A.2}
Similar to the first exercise, but now we have $C_1 C_2 h$ instead.

\subsubsection{Exercise A.3}
Logarithms of different bases differ by a constant.

\subsubsection{Exercise A.4}
The argument treats $k$ as a constant, when in reality $k = O(n)$. The answer should be
$O(n^3)$, which can be checked by looking at the closed form of the sum.

\subsubsection{Exercise A.5}
$2^{O(n)}$ and $O(2^{n})$ are not the same, since $\limsup_{n \to  \infty} \frac{2^{C_1n}}{C_{2}2^n}$
only converges to a nonzero constant when $C_1 = 1$.

\subsubsection{Exercise A.6}
$n^{k} \neq \Theta(2^n)$, $e^{-n} \neq \Theta(n^{-c})$, and $n! \neq \Theta(n^n)$, as all limits go to 0.

\subsubsection{Exercise A.7}
Take $f = n$ and $g = 2n$. 

\subsubsection{Exercise A.8}
1. We can pull out the exponents as constants, so $f = \Theta(g)$.

2. $3 < 2^2$ so $f = o(g)$.

3. $n = \omega(\log^2 n)$ so $f = \omega(g)$.

4. $2^{\log n} \to \infty$ so $f = \omega(g)$.

\subsubsection{Exercise A.9}
One example is $n^{\log n}$.
