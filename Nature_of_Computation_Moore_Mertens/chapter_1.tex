\section{Prologue}

\subsection{Exercises}

\subsubsection{Exercise 1}
The first graph has exactly 2 vertices of odd degree, so it has a Eulerian path. The second graph has 4 
vertices of odd degree, so it does not. 

\subsection{Problems}

\subsubsection{Problem 1}
Each edge in a finite graph increases the total sum of the graph's vertex degrees by 2. Thus, this sum must
be an even number, so we cannot have an odd number of vertices with odd degree (otherwise the sum would be odd).

\subsubsection{Problem 2}
Consider a finite simple graph with $n$ vertices. Each vertex can have a degree between 1 and $n - 1$, since
we cannot have multiple edges or self-loops. Thus, by the pigeonhole principle, at least two of the $n$ 
vertices must have the same degree (there is no bijection between the $n$ vertices and the $n - 1$ degree
options).

\subsubsection{Problem 3}
If every vertex has even degree, we can construct a set of covering cycles as follows. Choose an arbitrary
vertex with non-zero degree and traverse a cycle starting at that vertex, while deleting all edges traversed along the way.
Repeat this procedure until all edges have been deleted. We are guaranteed to be able to find the
aforementioned cycles since the graph is connected; an edge "leaving" a vertex can be paired with an edge
"coming in" to the vertex.

Once we have a set of covering cycles, we can combine them into a single cycle as follows. Consider a cycle
starting at vertex $a$ and another, edge-disjoint cycle starting at vertex $b$ such that $b$ also occurs
in the cycle starting at $a$. We can then combine the two cycles by starting at vertex $a$ and traversing
its cycle until arriving at vertex $b$, at which point we traverse the cycle starting at vertex $b$. Finally,
we finish the rest of vertex $a$'s cycle from $b$ onwards. Since cycles $a$ and $b$ were edge-disjoint, this
combined cycle does not visit any edge twice. We can repeat this cycle combination procedure until only a
single cycle is left, which is the Eulerian cycle.
