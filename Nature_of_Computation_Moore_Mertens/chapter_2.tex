\section{The Basics}

\subsection{Exercises}

\subsubsection{Exercise 2.1}
Let $a = nb + k$. If $n \geq 2$, then $\frac{a}{2} \geq b > a \mod b$ since $a \mod b \leq b-1$. For $n = 1$
we have
\begin{align*}
        k < b \implies \frac{k}{2} < \frac{b}{2} \implies k < \frac{b}{2} + \frac{k}{2} = \frac{a}{2}
\end{align*}
Since Euclid's algorithm ``alternates'' between performing divisions on $a$ and $b$ (since $a$ is replaced
by $b$ after a step), the number of divisions it performs is bounded by $2\log_2 a$.

\subsubsection{Exercise 2.2}
The maximum divisor of $a$ is bounded by $\sqrt{a}$. We can find all primes less than or equal to $\sqrt{a}$ 
in polynomial time by using a prime number sieve approach (e.g. Sieve of Eratosthenes). Once we have
the primes, the number of operations it takes to compute the prime factorization of $a$ is bounded by
$\sqrt{a} \log_2 a$, since the number of primes is bounded by $\sqrt{a}$ and  2 is the smallest prime.

\subsubsection{Exercise 2.3}
Changing logarithm base amounts to multiplying by a constant.

\subsubsection{Exercise 2.4}
We have the following
\begin{itemize}
        \item $4\log_2 n = \log_2 n^4 \implies n^4$
        \item $4\sqrt{n} = \sqrt{16n} \implies 16n$
        \item $4n \implies 4n$ 
        \item $4n^2 = (2n)^2 \implies 2n$
        \item $4 * 2^n = 2^{n+2} \implies n + 2$ 
        \item $4 * 4^n = 4^{n+1} \implies  n + 1$
\end{itemize}

\subsection{Problems}

\subsubsection{Problem 2.18}
An adjacency matrix requires a single bit for each vertex pair, so we need $n^2$ bits to specify it. A list
of edges consists of  $m$ tuples of the form $(a, b)$ for $1 \leq a, b \leq n$, so it requires $2m\log n$ 
bits. Thus, we are better off using the adjacency list format for sparse graphs and the adjacency matrix 
format for dense graphs. If, however, we consider multigraphs, then the adjacency matrix requires $n^2 \log n$
bits (as we need to store the number of edges from $a$ to $b$). Thus, in the multigraph case, we can always
choose to use adjacency lists.

\subsubsection{Problem 2.19}
Suppose $f(n) = O(n^a)$ and $g(n) = O(n^b)$. Then we have that
\begin{align*}
        \exists c_1, c_2, N \: | \: f(n) &\leq c_1 n^a, \: g(n) \leq c_2 n^b \quad \forall n \geq N\\
        \implies g(f(n)) &\leq g(c_1 n^a) \leq c_2 (c_1 n^a)^b = O(n^{ab})
\end{align*}
so $\text{poly}(n)$ is closed under composition. Thus, any polynomial time algorithm that calls another 
polynomial time algorithm remains polynomial.

\subsubsection{Problem 2.20}
Since $f(n) = 2^{\Theta(\log^k n)}$ and $g(n) = O(n^a)$, there exists $N$ such that for all $n \geq N$
we have $f(n) \geq 2^{c_1 \log^k n}$ and $g(n) \leq c_2 n^a$ for some constants $c_1, c_2$. Thus,
\begin{align*}
        \lim_{n \to \infty} \frac{f(n)}{g(n)} &\geq \lim_{n \to \infty} \frac{2^{c_1 \log^k n}}{c_2 n^a} \\
                                              &= \lim_{n \to  \infty} \frac{(n^{c_1})^{\log^{k - 1} n}}{c_2 n^a}
\end{align*}
Since $\lim_{n \to \infty}\log^{k - 1} n = \infty$, we have that $f(n) = \omega(g(n))$. To see that 
$f(n) = o(h(n))$, we substitute $n = 2^x$ to get
\begin{align*}
        \lim_{x \to \infty} \frac{f(2^x)}{g(2^x)} &\leq \lim_{x \to \infty} \frac{2^{a_1 x^k}}{2^{a_2 2^{c x}}} \\
                                                  &= 0
\end{align*}
Where we used the fact that $\lim_{x \to \infty} \frac{a_1 x^k}{a_2 2^{c x}} = 0$. Finally, to see that 
$\text{QuasiP}$ is closed under composition, we can check that 
\begin{align*}
        2^{\log^{k_1} 2^{\log^{k_2} n}} = 2^{\log^{k_1 k_2} n}
\end{align*}

