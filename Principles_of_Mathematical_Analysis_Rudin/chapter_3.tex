\section{Numerical Sequences and Series}

\subsection*{Definition 3.5}
Since $\{p_n\} \to p \implies \forall \epsilon, \: \exists N | n \geq N \implies \abs{p_n - p} < \epsilon$,
we can choose $k | n_k \geq N \implies \{p_{n_k}\} \to p$.
The reverse direction can be shown via contradiction of $\{p_n\} \to p$.

\subsection*{Examples 3.18}
(a) Density of rationals in reals.

(b) $\abs{s_n} < 1$, take $n$ odd to get -1 and even to get 1.

(c) Every subsequential limit has to converge to $s$.

\subsection*{Theorem 3.19}
For all $\{n_k\}$, we have $\exists K | k \geq K \implies  n_k \geq N \implies \lim_{k \to \infty} t_{n_k} - s_{n_k} \geq 0$.

\subsection*{Theorem 3.26}
$s_n = 1 + x + ... + x^n \implies x s_n = x + x^2 + ... + x^{n+1} \implies (1 - x) s_n = 1 - x^{n+1}$.

\subsection*{Examples 3.40}
(a) Root test: $n \to \infty$.

(b) Ratio test: $\frac{1}{n+1} \to 0$.

(c) $1 \to 1$.

(d) Ratio test: $\frac{n}{n+1} \to 1$. $z = 1$ leads to harmonic series.

(e) Ratio test: $\frac{n^2}{(n+1)^2} \to 1$.

\subsection*{Example 3.53}
$\sum_{k=1}^{\infty} \frac{1}{4k-3} + \frac{1}{4k-1} - \frac{1}{2k} < \frac{5}{6} + \sum_{k=2}^{\infty} \frac{1}{4k-4} + \frac{1}{4k-4} - \frac{1}{2k}$. The RHS converges since $\frac{1}{4k-4} + \frac{1}{4k-4} - \frac{1}{2k} = \frac{1}{2k^2 - 2k}$.

\newpage

\subsection{Exercise 1}
All we need is the inequality $\abs{s_n - s} \geq \abs{\abs{s_n} - \abs{s}}$.
The converse is not true, since we can take $s_n = (-1)^n$.

\subsection{Exercise 2}
My original idea: $\sqrt{(n + x)^2} - n = x$. Setting $(n + x)^2 \geq n^2 + n$ gives
$x^2 \geq (1 - 2x) n$. The last inequality is only true for all $n$ when $x \geq \frac{1}{2}$.
This implies that $\frac{1}{2}$ is the supremum of $\sqrt{n^2 + n} - n$. Since 
$\sqrt{n^2 + n} - n$ is increasing, it converges to $\frac{1}{2}$.

Better: $(\sqrt{n^2 + n} - n) (\sqrt{n^2 + n} + n) = n \implies \sqrt{n^2 + n} - n = \frac{1}{\sqrt{1 + \frac{1}{n}} + 1}$.

\subsection{Exercise 3}
Clearly $s_{n + 1} > s_n$. We can see that $s_n < 2$ by induction, since $s_1 < 2$ and
$2 + \sqrt{s_n} < 4$. This gives that $s_n$ is monotone and bounded, implying it converges.

\subsection{Exercise 4}
\begin{align*}
&s_{2m + 1} = \sum_{i = 1}^{m} (\frac{1}{2})^i, \: s_{2m} = \sum_{i = 2}^{m} (\frac{1}{2})^i \\
&\implies \limsup_{n \to \infty} s_n = 1, \: \liminf_{n \to \infty} s_n = \frac{1}{2}
\end{align*}

\subsection{Exercise 5}
\begin{align*}
        \limsup_{n \to \infty} (a_n + b_n) &= \sup_{\{k\}}{\{\lim_{k \to \infty} (a_{n_k} + b_{n_k})\}} \\
                                           &= \sup_{\{k\}}{\{\lim_{k \to \infty} a_{n_k} + \lim_{k \to \infty} b_{n_k}\}}
\end{align*}

\subsection{Exercise 6}
(a) $\sqrt{n + 1} - \sqrt{n} = \frac{1}{\sqrt{n + 1} + \sqrt{n}}$ diverges from comparison
to harmonic series (same technique as Exercise 2).

(b) Converges, by comparison to $\sum_{n = 1}^{\infty} \frac{1}{n^p}$ for $p = \frac{3}{2}$.

(c) Converges by root test, since $\lim_{n \to \infty} n^{\frac{1}{n}} = 1$.

(d) Converges when $\abs{z} > 1$ and diverges otherwise. To see this, put $z = \abs{z}e^{i\theta}$ 
to get $\lim_{n \to \infty} \frac{1}{1 + \abs{z}^n e^{ni\theta}}$.
