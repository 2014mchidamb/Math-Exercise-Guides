\documentclass{article}
\usepackage[utf8]{inputenc}
\usepackage{amsmath}
\usepackage{amssymb}
\usepackage{framed}
\usepackage[parfill]{parskip}
\usepackage{physics}

\begin{document}
\title{``Sparknotes" for \textit{Principles of Mathematical Analysis} by Walter Rudin}
\author{Muthu Chidambaram}
\date{Last Updated: May 28, 2019} 

\maketitle

\tableofcontents
\newpage 

\section*{About}

\begin{quote}
        \textit{``A modern mathematical proof is not very different from a modern machine,
or a modern test setup: the simple fundamental principles are hidden 
and almost invisible under a mass of technical details.''} - Hermann Weyl
\end{quote}

These notes contain short summaries of (my) proof ideas for exercises
and some theorems from the book \textit{Principles of Mathematical Analysis} by Walter Rudin.
I have tried to make the summaries as brief as possible, sometimes only one line or one equation.
My hope is that the summaries will give enough information to reconstruct a full proof without bogging the reader down with details. 
In many cases, I am sure that I inadvertently sacrificed clarity in an attempt to obtain brevity, and would greatly appreciate any feedback.

Also, I like when people include (what they presume to be) relevant quotes in their notes, so I have to ask you to forgive my haughtiness in starting these notes with a quote from Hermann Weyl.

\section{Some Point Set Topology}

\subsection{Naive Set Theory}

\subsubsection{Theorem 1}
We only prove (10) to illustrate the technique; the other parts are either immediate
from definitions or can be proved similarly.

(10) 
\begin{align*}
        &x \in \bigg(\cup_{S_i \in \mathcal{S}_1} S_i\bigg) \cap \bigg(\cup_{S_i \in \mathcal{S}_2} S_i\bigg) \\
        \implies &\exists i, j \: | \: x \in S_i \cap S_j \implies x \in \cup_{S_i \in \mathcal{S}_1, S_j \in \mathcal{S}_2} (S_i \cap S_j) \\
                 &x \in \cup_{S_i \in \mathcal{S}_1, S_j \in \mathcal{S}_2} (S_i \cap S_j) \implies \exists i, j \: | \: x \in S_i \cap S_j \\
        \implies &x \in \bigg(\cup_{S_i \in \mathcal{S}_1} S_i\bigg) \cap \bigg(\cup_{S_i \in \mathcal{S}_2} S_i\bigg)
\end{align*}


\section{Differentiation}

\subsection{Basic Definitions}

\subsubsection{Exercise 1}
Let the derivative of $f$ at $a$ be the linear operator $\lambda$. Then, by problem $1-10$, we have that
$\abs{\lambda(h)} \leq M \abs{h}$ for some $M \in \mathbb{R}$. Thus,
\begin{align*}
        \abs{(f(a + h) - f(a)} &= \frac{\abs{f(a + h) - f(a)}}{\abs{h}} * \abs{h} \\
                               &\leq \frac{\abs{f(a + h) - f(a) - \lambda(h)} + \abs{\lambda(h)}}{\abs{h}} * \abs{h} \\
        \implies \lim_{h \to 0} \abs{f(a + h) - f(a)} &\leq \lim_{h \to 0} \frac{\abs{f(a + h) - f(a) - \lambda(h)} + \abs{\lambda(h)}}{\abs{h}} \lim_{h \to 0} \abs{h} \\
                                                      &\leq M * 0 = 0
\end{align*}

\subsubsection{Exercise 2}
If $f$ is independent, we can define $g = f(x, y_0)$ for some arbitrary $y_0$. If there exists $g$ such that
$f(x, y) = g(x)$ for all $x, y$, then we have $f(x, y_1) = g(x) = f(x, y_2)$ so $f$ is independent of the
second variable. In this case, $f'(a, b) = g'(a)$.

\subsubsection{Exercise 3}
A function that is independent of both variables must, by construction, be constant.

\subsubsection{Exercise 4}
(a) When $t < 0$, we have
\begin{align*}
        h(t) = -t\abs{x} g\bigg(\frac{tx}{-t \abs{x}}\bigg) = t\abs{x}g\bigg(\frac{x}{\abs{x}}\bigg)
\end{align*}
which is the same as $h(t)$ when $t > 0$, so $h(t)$ is differentiable with derivative $f(x)$.

(b) Since $g(0, 1) = g(1, 0) = 0$, we have that $f(h, 0) = f(0, k) = 0$. Letting $\lambda = Df(0, 0)$, we have
that
\begin{align*}
        \lim_{(h, 0) \to 0} \frac{\abs{\lambda(h, 0)}}{\abs{h}} &= \lim_{(h, 0) \to 0} \frac{\abs{h} \lambda(1, 0)}{\abs{h}} \\
                                                                &= \lambda(1, 0) = 0
\end{align*}
if $\lambda$ exists. Similarly, considering $(0, k) \to 0$ gives $\lambda(0, 1) = 0$, so $\lambda = 0$. 
As a result, we have that $f$ is differentiable at $(0, 0)$ only if
\begin{align*}
        \lim_{(h, k) \to (0, 0)} \frac{\abs{f(h, k)}}{\abs{(h, k)}} = \lim_{(h, k) \to (0, 0)} \abs{g\bigg(\frac{(h, k)}{\abs{(h, k)}}\bigg)} = 0
\end{align*}
which implies that $g = 0$ (we can consider $t(x, y)$ as $t \to 0$ for every $(x, y)$ on the unit circle). 

\subsubsection{Exercise 5}
Apply the previous exercise with $g(x, y) = x\abs{y}$.

\subsubsection{Exercise 8}
If $f$ is differentiable, then
\begin{align*}
        \lim_{h \to 0} \frac{\abs{f(a + h) - f(a) - \lambda(h)}}{\abs{h}} &= 0 \\
                                                                          &= \lim_{h \to 0} \frac{\abs{(f_1(a + h) - f_1(a) - \lambda_1(h), f_2(a + h) - f_2(a) - \lambda_2(h))}}{\abs{h}} \\
                                                                          &\geq \lim_{h \to 0} \frac{\abs{f_1(a + h) - f_1(a) - \lambda_1(h)}}{\abs{h}}
\end{align*}
so $f_1$ and $f_2$ are both differentiable. The reverse direction is a straightforward application of the triangle
inequality.

\subsubsection{Exercise 9}
(a) If $f$ is differentiable at $a$, then we can let $g(x) = f(a) + f'(a) (x - a)$. For the other direction,
\begin{align*}
        \lim_{h \to 0} \frac{f(a + h) - a_0 - a_1h}{h} &= 0 \\
        \implies \lim_{h \to 0} \frac{f(a + h) - a_0}{h} &= a_1 \\
        \implies a_0 = f(a)
\end{align*}
so $f$ is differentiable at $a$.

(b) We can break up the $n^{\text{th}}$ order limit into
\begin{align*}
        \lim_{x \to  a} \frac{f(x) - g(x)}{(x - a)^n} &=
        \lim_{x \to a} \frac{f(x) - \sum_{i = 0}^{n - 1} \frac{f^{(i)}(a)}{i!} (x - a)^i}{(x - a)^n} +
        \frac{f^{(n)}(a)}{n!} \\
                                                      &= \frac{f^{(n)}(a)}{n!} - \frac{f^{(n)}(a)}{n!} = 0
\end{align*}
where the last step follows from repeated application of L'Hopital's.

\subsection{Basic Theorems}

\subsubsection{Exercise 11}
(a) Let $G$ be a function such that $G' = g$. Then
\begin{align*}
        f'(x, y) &= g(x + y) (\pi_1'(x, y) + \pi_2'(x, y)) \\
                 &= g(x + y) ((1, 0) + (0, 1)) = (g(x + y), g(x + y))
\end{align*}

(b) Following the setup of (a), we have
\begin{align*}
        f'(x, y) &= g(xy) (y \pi_1'(x, y) + x \pi_2'(x, y)) \\
                 &= g(xy) (y, x)
\end{align*}

\subsubsection{Exercise 12}
(a) Since $f(h, k) = hk f(1, 1)$, the desired result follows immediately from $\lim_{(h, k) \to 0} \frac{\abs{hk}}{\abs{(h, k)}} = 0$.

(b) We have that $f(a + h, b + k) - f(a, b) = f(a, k) + f(h, b) + f(h, k)$ from bilinearity, so $Df(a, b)(x, y) = f(a, y) + f(x, b)$ as
desired.

(c) The function $p: \mathbb{R} \times \mathbb{R} \to \mathbb{R}$ from Theorem 2-3 is bilinear, so it is just a special case
of (b).

\subsubsection{Exercise 14}
(a) We can bound the multilinear case with the bilinear case:
\begin{align*}
        \lim_{h \to 0} \frac{\abs{f(a_1, ..., h_i, ..., h_j, ..., a_k)}}{\abs{h}} &\leq
        \lim_{h \to 0} \frac{\abs{f(a_1, ..., h_i, ..., h_j, ..., a_k)}}{\abs{(h_i, h_j)}} \\ &\leq
        \lim_{(h_i, h_j) \to 0} \frac{\abs{f(a_1, ..., h_i, ..., h_j, ..., a_k)}}{\abs{(h_i, h_j)}} \\
                                                                                             &= 0
\end{align*}

(b) Expanding $f(a_1 + h_1, ..., a_n + h_n) - f(a_1, ..., a_n) - Df(a_1, ..., a_n)(h_1, ..., h_n)$
leaves only terms that are at least bilinear, so by part (a) we are done.

\subsubsection{Exercise 16}
Using the fact that $Dx = x$ (from Theorem 2-3), we have that
\begin{align*}
        f'(f^{-1}) \circ f^{-1\prime} (x) &= x \\
        \implies f^{-1\prime} (x) &= [f'(f^{-1} (x))]^{-1} 
\end{align*}

\section{Insights and Algorithms}

\subsection{Exercises}

\subsubsection{Exercise 3.1}
Assume $f(n) = 2^{n} - 1$. Then $f(n + 1) = 2^{n + 1} - 2 + 1 = 2^{n + 1} - 1$.

\subsubsection{Exercise 3.2}
We have that
\begin{align*}
        (QQ^* )_{ab} &= \frac{1}{n}\sum_{k = 0}^{n - 1} w_n^{ak} \bar{w_n^{kb}} \\
                     &= \frac{1}{n}\sum_{k = 0}^{n - 1} w_n^{k(i - j)}\\
\end{align*}
If $i = j$, then $(QQ^*)_{ab} = 1$. Otherwise, since $w_n^{(i - j)}$ is a root of unity,
\begin{align*}
        (QQ^*)_{ab} &= w_n^{i - j} (QQ^*)_{ab} \implies (1 - w_n^{i - j}) (QQ^*)_{ab} = 0 \implies
        (QQ^*)_{ab} = 0
\end{align*}
And we have that $Q^* = Q^{-1}$ as desired.

\subsubsection{Exercise 3.3}
Exercise seems ambiguous - if we're just talking about $f(1)$, then we can simply add an $if$ case for it.
Otherwise, we can return both $f_min$ and the index $j$. I don't think it's possible to 
reconstruct full scores from just returning the index by itself (without wasting computation).

\subsubsection{Exercise 3.4}
If we let $g(n)$ be the number of calculations for $f(n)$ after calling $f(1)$, then we have by definition
of the algorithm that $g(n) = \sum_{k = 1}^{n - 1} g(k)$ (since every call to $f(k)$ calls $f(n)$).
We see that $g(2) = g(1) = 2^0$, and we assume that $g(n) = 2^{n - 2}$. Then we have that
\begin{align*}
        g(n + 1) = \sum_{k = 1}^{n} g(k) = g(n) + \sum_{k = 1}^{n - 1} g(k) = 2g(n) = 2^{n - 1}
\end{align*}
So we are done by induction.

\subsubsection{Exercise 3.5}
Suppose we pick a subset of $j$ characters from both $s$ and $t$. Then there is a unique alignment that 
corresponds to assigning the subsets to one another (in the order of characters) and then deleting/inserting
everything else that's not aligned. This also covers all possible alignments, so we have that the number of
alignments is $\binom{2n}{n}$.

\subsubsection{Exercise 3.6}
There are $n$ choices for where to cut $s$ to make $s'$ and $n$ choices for where to cut $t$ to make $t'$,
hence $O(n^2)$.

\subsubsection{Exercise 3.7}
Calculating optimal edit distance finds an optimal alignment in the process; we only need to also return
with $d(s, t)$ the choice of operation that was used and then compose these operations across subproblems.

The edit distance problem is available to solve on 
\href{https://leetcode.com/problems/edit-distance/}{LeetCode}. 
My solution is as follows
\begin{lstlisting}[language=python]
    def minDistance(self, word1, word2):
        N, M = len(word1), len(word2)
        DP = [[0] * (M+1) for i in range(N+1)]
        for j in range(M):
            DP[N][j] = M - j
        for i in range(N):
            DP[i][M] = N - i
        for i in range(N-1, -1, -1):
            for j in range(M-1, -1, -1):
                not_eq = 1
                if word1[i] == word2[j]:
                    not_eq = 0
                DP[i][j] = min(DP[i][j+1] + 1,
                               DP[i+1][j] + 1,
                               DP[i+1][j+1] + not_eq)
                
        return DP[0][0]
\end{lstlisting}

\subsubsection{Exercise 3.8}
If there is a path $s \to t$, then there must be a product of the form $A_si A_ij ... A_kt$ that is non-zero 
and contains a maximum of $n-1$ terms. Any such product can be extended to $n-1$ terms if we introduce
self-loops (since $A_ii = 1$), so we have that $(1 + A)_{st}^{n-1}$ is non-zero. The reverse direction can
be shown similarly.

\subsubsection{Exercise 3.9}
As hinted, we can maintain an array $V[i][j]$ that tracks which middle vertex $k$ was used to get the minimum 
$B_{ij}(\log_2 n)$. Then, we can reconstruct the optimal path backwards by looking at
$V[i][j] = k_1, V[i][k_1] = k_2, ...$ until we get to $i$.

\subsubsection{Exercise 3.10}
If there is a cycle whose total length is negative, then there is no fixed point (since we can repeatedly go
through that cycle to decrease distance). Otherwise, there is no issue, since there is no way to reduce
distance by visiting a vertex more than once.

\subsubsection{Exercise 3.11}
By definition, $B_{ij}(m)$ must be an upper bound on the length of the shortest path from $i$ to $j$, as 
otherwise we would be positing that there exists $k$ such that the shortest path from $i$ to $k$ to $j$ is
shorter than the shortest path from  $i$ to  $j$. The term $B_{ij}(m)$ is the composition of paths 
$B{ik}(m-1)$ and  $B_{kj}(m-1)$, so the number of steps in $B_{ij}(m)$ is at most twice the maximum of
the number of steps in $B_{ik}(m-1)$ (over all $k$). From this it is clear that the number of steps after
$m$ outermost iterations is at most $2^m$ (since $m = 0$ corresponds to a single step). If there are no
negative cycles, then the shortest path between two vertices will take at most $n-1$ steps, so we only need
to iterate up to $m = \log_2 m$.


\end{document}
