\documentclass{article}
\usepackage[utf8]{inputenc}
\usepackage{amsmath}
\usepackage{amssymb}
\usepackage{framed}
\usepackage[parfill]{parskip}
\usepackage{physics}

\begin{document}
\title{``Sparknotes" for \textit{Principles of Mathematical Analysis} by Walter Rudin}
\author{Muthu Chidambaram}
\date{Last Updated: May 28, 2019} 

\maketitle

\tableofcontents
\newpage 

\section*{About}

\begin{quote}
        \textit{``A modern mathematical proof is not very different from a modern machine,
or a modern test setup: the simple fundamental principles are hidden 
and almost invisible under a mass of technical details.''} - Hermann Weyl
\end{quote}

These notes contain short summaries of (my) proof ideas for exercises
and some theorems from the book \textit{Principles of Mathematical Analysis} by Walter Rudin.
I have tried to make the summaries as brief as possible, sometimes only one line or one equation.
My hope is that the summaries will give enough information to reconstruct a full proof without bogging the reader down with details. 
In many cases, I am sure that I inadvertently sacrificed clarity in an attempt to obtain brevity, and would greatly appreciate any feedback.

Also, I like when people include (what they presume to be) relevant quotes in their notes, so I have to ask you to forgive my haughtiness in starting these notes with a quote from Hermann Weyl.

\section{Some Point Set Topology}

\subsection{Naive Set Theory}

\subsubsection{Theorem 1}
We only prove (10) to illustrate the technique; the other parts are either immediate
from definitions or can be proved similarly.

(10) 
\begin{align*}
        &x \in \bigg(\cup_{S_i \in \mathcal{S}_1} S_i\bigg) \cap \bigg(\cup_{S_i \in \mathcal{S}_2} S_i\bigg) \\
        \implies &\exists i, j \: | \: x \in S_i \cap S_j \implies x \in \cup_{S_i \in \mathcal{S}_1, S_j \in \mathcal{S}_2} (S_i \cap S_j) \\
                 &x \in \cup_{S_i \in \mathcal{S}_1, S_j \in \mathcal{S}_2} (S_i \cap S_j) \implies \exists i, j \: | \: x \in S_i \cap S_j \\
        \implies &x \in \bigg(\cup_{S_i \in \mathcal{S}_1} S_i\bigg) \cap \bigg(\cup_{S_i \in \mathcal{S}_2} S_i\bigg)
\end{align*}


\section{Numerical Sequences and Series}

\subsection*{Definition 3.5}
Since $\{p_n\} \to p \implies \forall \epsilon, \: \exists N | n \geq N \implies \abs{p_n - p} < \epsilon$,
we can choose $k | n_k \geq N \implies \{p_{n_k}\} \to p$.
The reverse direction can be shown via contradiction of $\{p_n\} \to p$.

\subsection*{Examples 3.18}
(a) Density of rationals in reals.

(b) $\abs{s_n} < 1$, take $n$ odd to get -1 and even to get 1.

(c) Every subsequential limit has to converge to $s$.

\subsection*{Theorem 3.19}
For all $\{n_k\}$, we have $\exists K | k \geq K \implies  n_k \geq N \implies \lim_{k \to \infty} t_{n_k} - s_{n_k} \geq 0$.

\subsection*{Theorem 3.26}
$s_n = 1 + x + ... + x^n \implies x s_n = x + x^2 + ... + x^{n+1} \implies (1 - x) s_n = 1 - x^{n+1}$.


\end{document}
