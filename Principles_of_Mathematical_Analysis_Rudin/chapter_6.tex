\section{The Riemann-Stieltjes Integral}

\subsection{Exercise 1}
We first note that $\inf{f(x)} = 0$ on any interval $ [p_{i-1}, p_i] \subset [a, b]$. Similarly,
$\sup{f(x)} = 0$ if $x_0 \notin [p_{i-1}, p_i]$ and 1 otherwise. Thus, we proceed by constructing a
partition $P$ of $[a, b]$ such that $x_0 \in [p_{i-1}, p_i]$ and $\alpha(p_i) - \alpha(p_{i-1}) < \epsilon$.
This is possible since $\alpha$ is continuous at $x_0$, so there exists $\delta$ such that choosing
$p_i - p_{i-1} < \delta$ gives us the previous inequality. We then have that  
$U(P, f, \alpha) - L(P, f, \alpha) < \epsilon$, so $f \in \mathscr{R}(\alpha)$. Furthermore, since
$L(P, f, \alpha) = 0$ for all $P$, we have that $\int_{a}^{b} f d\alpha = 0$.

\subsection{Exercise 2}
Since we're given $f(x) \geq 0$ and $\int_{a}^{b} f(x) dx = 0$, we have that $m = \inf{f(x)} = 0$. Letting
$m_i$ and $M_i$ denote the infimum and supremum of $f$ on the interval $[x_{i-1}, x_i] \subset [a, b]$,
we further have that $m_i \leq m \implies m_i = 0 \: \forall i$. Additionally, since $f$ is continuous
on $[a, b]$ and therefore uniformly continuous (from compactness), we can choose a $\delta$ such that
\begin{align*}
        \abs{x_i - x_{i-1}} < \delta &\implies \abs{f(x_i) - f(x_{i-1})} < \epsilon \\
                                     &\implies \abs{M_i - m_i} < \epsilon \implies M_i < \epsilon
\end{align*}
Thus, all of the $M_i$ can be made arbitrarily small, implying that $\sup{f(x)} < \epsilon$ for every 
positive $\epsilon$. This gives us that $f(x) = 0$.

\subsection{Exercise 3}
(a) Suppose $f(0+) = f(0)$. We consider the partition $P = {x_0, x_1, x_2, x_3}$ where  
$x_0 = -1, x_1 = 0, x_3 = 1$. Then $U(P, f, \beta_1) = M_2$ and $L(P, f, \beta_1) = m_2$, and
we have that $M_2 \to f(0)$ and $m_2 \to f(0)$ as $x_2 \to 0$, so $f \in \mathscr{R}(\beta_1)$.
For the other direction, if $f \in \mathscr{R}(\beta_1)$, then there exists $P$ such that 
$U(P, f, \beta_1) - L(P, f, \beta_1) < \epsilon$. We can then consider the refinement $P^*$ of $P$ 
that contains an interval of the form $[0, x_i]$. Suppose now that $\abs{f(0+) - f(0)} = \epsilon > 0$.
Then
\begin{align*}
        M_i - m_i \geq \abs{f(0+) - f(0)} > \epsilon
\end{align*}
which is a contradiction, so $f(0+) = f(0)$.

(b) Only difference from (a) is that we need $f(0-) = f(0)$ instead of $f(0+) = f(0)$. The only changes that
need to be made to the proof of (a) involve replacing $[0, x_i]$ with $[x_{i-1}, 0]$.

(c) If $f$ is continuous at 0, we can consider the partition $P$ that contains $x_{i-1} < x_i = 0 < x_{i+1}$.
Then $U(P, f, \beta_2) = \frac{M_{i} + M_{i+1}}{2}$ and $L(P, f, \beta_2) = \frac{m_i + m_{i+1}}{2}$.
Since $f$ is continuous at 0, $M_i, M_{i+1}, m_i, m_{i+1} \to f(0)$ as $x_{i-1} \to 0$ and $x_{i+1} \to 0$,
so $f \in \mathscr{R}(\beta_2)$.
For the other direction, we proceed similarly to the proof of (a) and see that
\begin{align*}
        \frac{1}{2} (M_i + M_{i + 1}) - \frac{1}{2} (m_i + m_{i + 1}) &= \frac{1}{2} ((M_i - m_i) + (M_{i+1} - m_{i+1})) \\
                                                                      &\geq \frac{1}{2} (\abs{f(0-) - f(0)} + \abs{f(0+) - f(0)}) \\
                                                                      &> \epsilon
\end{align*}
for some $\epsilon > 0$ unless $f(0-) = f(0+) = f(0)$, so  $f$ must be continuous at 0 if  $f \in \mathscr{R}(\beta_2)$.

(d) If $f$ is continuous at 0 then we have $f(0+) = f(0-) = f(0)$, so we are done by parts (a)-(c).

\subsection{Exercise 4}
From the density of the rationals and irrationals in the reals, we have that $M_i - m_i = 1 \: \forall i$, so
$f \notin  \mathscr{R}$.

\subsection{Exercise 5}
If we consider $f(x) = 1$ for all rational $x$ and $f(x) = -1$ for all irrational $x$, we have that
$f^2 \in \mathscr{R}$ but $f \notin \mathscr{R}$. However, if $f^3 \in \mathscr{R}$, then $f \in \mathscr{R}$.
This is because $m \leq f^3 \leq M$ (since $f$ is bounded) and $x^{\frac{1}{3}}$ is continuous on any
$[m, M]$, so we can apply Theorem 6.11.

\subsection{Exercise 6}

