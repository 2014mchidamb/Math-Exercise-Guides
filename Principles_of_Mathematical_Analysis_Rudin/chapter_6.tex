\section{The Riemann-Stieltjes Integral}

\subsection{Exercise 1}
We first note that $\inf{f(x)} = 0$ on any interval $ [p_{i-1}, p_i] \subset [a, b]$. Similarly,
$\sup{f(x)} = 0$ if $x_0 \notin [p_{i-1}, p_i]$ and 1 otherwise. Thus, we proceed by constructing a
partition $P$ of $[a, b]$ such that $x_0 \in [p_{i-1}, p_i]$ and $\alpha(p_i) - \alpha(p_{i-1}) < \epsilon$.
This is possible since $\alpha$ is continuous at $x_0$, so there exists $\delta$ such that choosing
$p_i - p_{i-1} < \delta$ gives us the previous inequality. We then have that  
$U(P, f, \alpha) - L(P, f, \alpha) < \epsilon$, so $f \in \mathscr{R}(\alpha)$. Furthermore, since
$L(P, f, \alpha) = 0$ for all $P$, we have that $\int_{a}^{b} f d\alpha = 0$.

\subsection{Exercise 2}
Since we're given $f(x) \geq 0$ and $\int_{a}^{b} f(x) dx = 0$, we have that $m = \inf{f(x)} = 0$. Letting
$m_i$ and $M_i$ denote the infimum and supremum of $f$ on the interval $[x_{i-1}, x_i] \subset [a, b]$,
we further have that $m_i \leq m \implies m_i = 0 \: \forall i$. Additionally, since $f$ is continuous
on $[a, b]$ and therefore uniformly continuous (from compactness), we can choose a $\delta$ such that
\begin{align*}
        \abs{x_i - x_{i-1}} < \delta &\implies \abs{f(x_i) - f(x_{i-1})} < \epsilon \\
                                     &\implies \abs{M_i - m_i} < \epsilon \implies M_i < \epsilon
\end{align*}
Thus, all of the $M_i$ can be made arbitrarily small, implying that $\sup{f(x)} < \epsilon$ for every 
positive $\epsilon$. This gives us that $f(x) = 0$.

\subsection{Exercise 3}
(a) Suppose $f(0+) = f(0)$. We consider the partition $P = {x_0, x_1, x_2, x_3}$ where  
$x_0 = -1, x_1 = 0, x_3 = 1$. Then $U(P, f, \beta_1) = M_2$ and $L(P, f, \beta_1) = m_2$, and
we have that $M_2 \to f(0)$ and $m_2 \to f(0)$ as $x_2 \to 0$, so $f \in \mathscr{R}(\beta_1)$.
For the other direction, if $f \in \mathscr{R}(\beta_1)$, then there exists $P$ such that 
$U(P, f, \beta_1) - L(P, f, \beta_1) < \epsilon$. We can then consider the refinement $P^*$ of $P$ 
that contains an interval of the form $[0, x_i]$. Suppose now that $\abs{f(0+) - f(0)} = \epsilon > 0$.
Then
\begin{align*}
        M_i - m_i \geq \abs{f(0+) - f(0)} > \epsilon
\end{align*}
which is a contradiction, so $f(0+) = f(0)$.

(b) Only difference from (a) is that we need $f(0-) = f(0)$ instead of $f(0+) = f(0)$. The only changes that
need to be made to the proof of (a) involve replacing $[0, x_i]$ with $[x_{i-1}, 0]$.

(c) If $f$ is continuous at 0, we can consider the partition $P$ that contains $x_{i-1} < x_i = 0 < x_{i+1}$.
Then $U(P, f, \beta_2) = \frac{M_{i} + M_{i+1}}{2}$ and $L(P, f, \beta_2) = \frac{m_i + m_{i+1}}{2}$.
Since $f$ is continuous at 0, $M_i, M_{i+1}, m_i, m_{i+1} \to f(0)$ as $x_{i-1} \to 0$ and $x_{i+1} \to 0$,
so $f \in \mathscr{R}(\beta_2)$.
For the other direction, we proceed similarly to the proof of (a) and see that
\begin{align*}
        \frac{1}{2} (M_i + M_{i + 1}) - \frac{1}{2} (m_i + m_{i + 1}) &= \frac{1}{2} ((M_i - m_i) + (M_{i+1} - m_{i+1})) \\
                                                                      &\geq \frac{1}{2} (\abs{f(0-) - f(0)} + \abs{f(0+) - f(0)}) \\
                                                                      &> \epsilon
\end{align*}
for some $\epsilon > 0$ unless $f(0-) = f(0+) = f(0)$, so  $f$ must be continuous at 0 if  $f \in \mathscr{R}(\beta_2)$.

(d) If $f$ is continuous at 0 then we have $f(0+) = f(0-) = f(0)$, so we are done by parts (a)-(c).

\subsection{Exercise 4}
From the density of the rationals and irrationals in the reals, we have that $M_i - m_i = 1 \: \forall i$, so
$f \notin  \mathscr{R}$.

\subsection{Exercise 5}
If we consider $f(x) = 1$ for all rational $x$ and $f(x) = -1$ for all irrational $x$, we have that
$f^2 \in \mathscr{R}$ but $f \notin \mathscr{R}$. However, if $f^3 \in \mathscr{R}$, then $f \in \mathscr{R}$.
This is because $m \leq f^3 \leq M$ (since $f$ is bounded) and $x^{\frac{1}{3}}$ is continuous on any
$[m, M]$, so we can apply Theorem 6.11.

\subsection{Exercise 6}
$P$ is compact, so the open cover consisting of neighborhoods around each point of $P$ has a finite subcover.
Thus, $P$ can be covered by finitely many segments. Additionally, the total length of these segments can be
made arbitrarily small, since $P$ contains no segments itself. We can then proceed exactly as in the proof of
Theorem 6.10 to get the desired result.

\subsection{Exercise 7}
(a) If $f \in \mathscr{R}$, then by Theorem 6.12 (c) we have
\begin{align*}
        \abs{\int_{c}^{1} f dx - \int_{0}^{1} f dx} &= \abs{\int_{0}^{c} f dx} \\
                                                    &\leq cM < \epsilon \quad \text{for} \:\:  c < \frac{\epsilon}{M}
\end{align*}
Where $M = \sup{f}$ on $[0, 1]$.

(b) Consider $f$ such that
\begin{align*}
        f(x) =
        \begin{cases}
                0 & x = 0 \\
                \frac{1}{x} & 0 < x < \frac{1}{2} \\
                0 & x = \frac{1}{2} \\
                -\frac{1}{x} & \frac{1}{2} < x \leq 1
        \end{cases}
\end{align*}
Then $\int_{0}^{1} f dx = 0$, but $\int_{0}^{1} \abs{f} dx$ diverges.

\subsection{Exercise 8}
We see that the partition $P = {1, 2, ..., b + 1}$ of the interval $[1, b + 1]$ corresponds to the upper sum
$U(P, f) = \sum_{n = 1}^b f(n)$, and $\int_{1}^{b + 1} f(x) dx \leq U(P, f)$. 
Thus, if $\sum_{n = 1}^\infty f(n)$ converges, then we have that 
$\int_{1}^{b + 1} f(x) dx$ is bounded and monotonically increasing (since $f(x) \geq 0$),
so $\int_{1}^{\infty} f(x) dx $ converges. For the other direction, we can take the same partition $P$ 
and consider $L(P, f) = \sum_{n = 1}^b f(n + 1)$. Since $L(P, f) \leq \int_{1}^{b + 1} f(x) dx$, if 
$\int_{1}^{\infty} f(x) dx$ converges then so does $\sum_{n = 2}^\infty f(n)$, and we are done (as
$f(1) < \infty$).

\subsection{Exercise 9}
Theorem: Let $f$ and $g$ be differentiable functions on $[0, \infty)$ such that 
$\lim_{x \to \infty} f(x)g(x) = 0$ and $f', g' \in \mathscr{R}$ for every $[0, b]$. Then we have that
\begin{align*}
        \int_{0}^{\infty} f(x) g'(x) dx = -f(0)g(0) - \int_{0}^{\infty} f'(x) g(x) 
\end{align*}
if both improper integrals converge.

Proof: Take limits on both sides of the integration by parts formula.

Letting $f(x) = \frac{1}{1 + x}$ and $g'(x) = cos(x)$ in the above formula shows that 
\begin{align*}
        \int_{0}^{\infty} \frac{cos(x)}{1 + x} dx = \int_{0}^{\infty} \frac{sin(x)}{(1 + x)^2} dx  
\end{align*}

We have that $\abs{cos(n)} + \abs{cos(n + 1)} \geq c$ for some
constant $c$ since $cos(n)$ and $cos(n + 1)$ cannot both be 0. As such, 
$\sum_{n = 1}^\infty \frac{cos(n)}{1 + n}$ diverges since the harmonic series diverges and therefore
the integral on the left also diverges by the integral test.

\subsection{Exercise 10}
(a) We can use the convexity of $e^x$ to show this. Let $\lambda = \frac{1}{p}$ and let $u^p = e^a, v^q = e^b$
for some $a, b$ (this is possible because $u, v \geq 0$). Then we have
\begin{align*}
        e^{\lambda a + (1- \lambda) b} &\leq \lambda u^p + (1 - \lambda) v^q \\
        e^{\lambda a} e^{(1 - \lambda) b} &\leq \frac{u^p}{p} + \frac{v^q}{q} \\
        uv &\leq \frac{u^p}{p} + \frac{v^q}{q}
\end{align*}

(b) From part (a) we have that 
\begin{align*}
        f(x) g(x) &\leq \frac{f^p(x)}{p} + \frac{g^q(x)}{q} \\
        \int_{a}^{b} fg d\alpha &\leq 1 
\end{align*}

(c) We can use part (a) with  $u = \frac{\abs{f(x)}}{\bigg(\int_{a}^{b} f^p d\alpha\bigg)^{\frac{1}{p}}}$ 
and $v = \frac{\abs{g(x)}}{\bigg(\int_{a}^{b} g^q d\alpha\bigg)^{\frac{1}{q}}}$ to get the desired result.

(d) Assuming the limits exist, the inequality follows for impromper integrals by taking limits on both sides
and then using limit rules.
