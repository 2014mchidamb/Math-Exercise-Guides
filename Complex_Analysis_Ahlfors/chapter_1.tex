\section{Complex Numbers}

\subsection{Arithmetic Operations}

\subsubsection{Exercise 3}
This exercise is much easier if we represent the complex numbers in polar form (which has not been introduced
yet). We see that
\begin{align*}
        \frac{-1 \pm i \sqrt{3}}{2} &= e^{i \frac{2\pi}{3}}, \: e^{i \frac{4 \pi}{3}} \\
        \frac{\pm 1 \pm i \sqrt{3}}{2} &= e^{i \frac{2\pi}{3}}, \: e^{i \frac{4 \pi}{3}}, \:  
        e^{i \frac{\pi}{3}}, \: e^{i \frac{5 \pi}{3}} \\
\end{align*}
which gives us the desired equality since $e^{i 2\pi} = 1$.

\subsection{Square Roots}
Once again, all of these computational exercises are made much easier with polar form. 
They seem more tedious than instructive.

\subsection{Justification}
Going to come back to this after finishing chapter 3 of Birkhoff/MacLane (which I've been neglecting...).

\subsection{Conjugation, Absolute Value}

\subsubsection{Exercise 3}
We can manipulate the equality to get
\begin{align*}
        \norm{\frac{a - b}{1 - \bar{a}b}} &= 1 \\
        \norm{a - b}^2 &= \norm{1 - \bar{a}b}^2 \\
        \norm{a}^2 + \norm{b}^2 - 2\Re(a\bar{b}) &= 1 + \norm{a}^2 \norm{b}^2 - 2\Re(a\bar{b}) 
\end{align*}
Thus we see that equality holds if either $\norm{a} = 1$ or $\norm{b} = 1$, excepting the case where 
$a = b = 1$ as that makes the denominator in the equality 0.

\subsubsection{Exercise 4}
Let $z = \alpha + \beta i$. Then we have
\begin{align*}
        \big((a + b) \alpha + c\big) + (a - b)\beta i &= 0 \\
        \implies (a + b) \alpha + c &= 0 \\
        \implies (a - b) \beta = 0
\end{align*}
If $a - b = 0$, $\beta$ could be anything, so we must have $a - b \neq 0$ for the solution to be unique.
Similarly, if $a + b = 0$ then $\alpha$ can either be anything or there is no solution for $\alpha$ 
(if $c \neq 0$). Thus, the two conditions we need are  $a + b \neq 0$ and $a - b \neq 0$.

\subsubsection{Exercise 5}

