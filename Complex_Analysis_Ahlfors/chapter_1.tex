\section{Complex Numbers}

\subsection{Arithmetic Operations}

\subsubsection{Exercise 3}
This exercise is much easier if we represent the complex numbers in polar form (which has not been introduced
yet). We see that
\begin{align*}
        \frac{-1 \pm i \sqrt{3}}{2} &= e^{i \frac{2\pi}{3}}, \: e^{i \frac{4 \pi}{3}} \\
        \frac{\pm 1 \pm i \sqrt{3}}{2} &= e^{i \frac{2\pi}{3}}, \: e^{i \frac{4 \pi}{3}}, \:  
        e^{i \frac{\pi}{3}}, \: e^{i \frac{5 \pi}{3}} \\
\end{align*}
which gives us the desired equality since $e^{i 2\pi} = 1$.

\subsection{Square Roots}
Once again, all of these computational exercises are made much easier with polar form. 
They seem more tedious than instructive.

\subsection{Justification}
Going to come back to this after finishing chapter 3 of Birkhoff/MacLane (which I've been neglecting...).

\subsection{Conjugation, Absolute Value}

\subsubsection{Exercise 3}
We can manipulate the equality to get
\begin{align*}
        \norm{\frac{a - b}{1 - \bar{a}b}} &= 1 \\
        \norm{a - b}^2 &= \norm{1 - \bar{a}b}^2 \\
        \norm{a}^2 + \norm{b}^2 - 2\Re(a\bar{b}) &= 1 + \norm{a}^2 \norm{b}^2 - 2\Re(a\bar{b}) 
\end{align*}
Thus we see that equality holds if either $\norm{a} = 1$ or $\norm{b} = 1$, excepting the case where 
$a = b = 1$ as that makes the denominator in the equality 0.

\subsubsection{Exercise 4}
Let $z = \alpha + \beta i$. Then we have
\begin{align*}
        \big((a + b) \alpha + c\big) + (a - b)\beta i &= 0 \\
        \implies (a + b) \alpha + c &= 0 \\
        \implies (a - b) \beta = 0
\end{align*}
If $a - b = 0$, $\beta$ could be anything, so we must have $a - b \neq 0$ for the solution to be unique.
Similarly, if $a + b = 0$ then $\alpha$ can either be anything or there is no solution for $\alpha$ 
(if $c \neq 0$). Thus, the two conditions we need are  $a + b \neq 0$ and $a - b \neq 0$.

\subsubsection{Exercise 5}
We can write $\abs{\sum_{i = 1}^n a_i b_i}^2$ as $(\sum_{i = 1}^n a_i b_i) (\sum_{j = 1}^n \overline{a_j b_j})$
to see that it can be expanded as a sum whose terms consist of $\abs{a_i}^2 \abs{b_i}^2$ and 
$a_i b_i \overline{a_j b_j}$ over all $1 \leq i, j \leq n$. The $a_i b_i \overline{a_j b_j}$ terms can be
paired with the $a_j b_j \overline{a_i b_i}$ terms to get that
\begin{align*}
        \abs{\sum_{i = 1}^n a_i b_i}^2 &= \sum_{i = 1}^n \abs{a_i}^2 \abs{b_i}^2 + \sum_{1 \leq i < j \leq n} 2 \Re a_i b_i \overline{a_j b_j} \\
                                       &= \sum_{i = 1}^n \abs{a_i}^2 \sum_{i = 1}^n \abs{b_i}^2 - \sum_{1 \leq i < j \leq n} \abs{a_i \bar{b}_j - a_j \bar{b}_i}^2
\end{align*}

\subsection{Inequalities}

\subsubsection{Exercise 3}
We apply the triangle inequality and then Cauchy-Schwarz to get
\begin{align*}
        \abs{\sum_{i} \lambda_i a_i} \leq \sum_{i} \abs{\lambda_i a_i} \leq \sum_{i} \abs{\lambda_i} \abs{a_i} < \sum_{i} \abs{\lambda_i} = 1
\end{align*}
Where the final strict inequality follows from $\abs{a_i} < 1$.

\subsubsection{Exercise 4}
We have that $\abs{z - a} = \abs{z} - \abs{a}$ when $\frac{z}{a} \geq 1$ (applying our criterion for equality
to $\abs{(z - a) + a}$). Thus, if $\abs{a} \leq \abs{c}$, we can choose $z$ such that $\frac{z}{a} \geq 1$ 
and $\abs{z} = \abs{c}$, so a solution to the equation exists. Now suppose a solution exists to 
 $\abs{z - a} + \abs{z + a} = 2\abs{c}$. Then by applying the triangle inequality, we have 
 \begin{align*}
         2\abs{c} \geq \abs{z - a} + \abs{z + a} \geq \abs{z - a - z - a} = 2\abs{a}
 \end{align*}
 so $\abs{a} \leq \abs{c}$.

\subsection{Geometric Addition and Multiplication}

\subsubsection{Exercise 2}
We will only show one direction, since the other direction just reverses the steps. Suppose that 
$a_1, a_2, a_3$ indicate the vertices of an equilateral triangle. Then the angles between the edges
$a_1 - a_3, a_3 - a_2, a_2 - a_1$ must be equal. Using the fact that $\text{arg} \frac{a_2}{a_1} = \text{arg} a_2 - \text{arg} a_1$, we get
\begin{align*}
        \frac{a_1 - a_3}{a_3 - a_2} &= \frac{a_3 - a_2}{a_2 - a_1} \\
        a_1 a_2 - a_1^2 - a_2 a_3 + a_3 a_1 &= a_3^2 - 2 a_2 a_3 + a_2^2 \\
        a_1 a_2 + a_2 a_3 + a_3 a_1 &= a_1^2 + a_2^2 + a_3^2
\end{align*}

\subsubsection{Exercise 4}
Let the center of the circle be $c$. Then we can proceed similarly to Exercise 2 by noting that since
$\abs{c - a_1} = \abs{c - a_2} = \abs{c - a_3}$, we have that the angle between $c - a_1$ and $a_1 - a_3$ 
is the same as the angle between $a_1 - a_3$ and $a_3 - c$. We can then solve for $c$ and use $c$ to compute
the radius. The result, though, is messy and not fun to typeset.

\subsection{The Binomial Equation}

\subsubsection{Exercise 4}
Let $s = 1 + w^h + ... + w^{(n - 1)h}$. Then we have that $w^h s = w^h + w^{2h} + ... + w^{nh}$, which implies
that $w^h s = s$ since $w^{nh} = 1$. If $h$ is not a multiple of $n$, then $w^h \neq 1$, so $s = 0$.

\subsubsection{Exercise 5}
The approach is the same as Exercise 4. Letting $s$ denote the sum, we multiply by $-w^h$ to get that
$-w^h s = (-1)^n - w^h + ... + (-1)^{n - 1} w^{(n - 1)h}$. Thus, $s = 0$ if $n$ is even (even if $w^h = -1$).
Otherwise, we solve $-w^h s = s - 2$ to get that $s = \frac{2}{1 + w^h}$ ($w^h \neq -1$ since $n$ is odd).

\subsection{Analytic Geometry}

\subsubsection{Exercise 1}
We refer back to Exercise 4 from the section ``Conjugation, Absolute Value''. If $z = \alpha + \beta i$, then
\begin{align*}
        (a + b) \alpha + c &= 0 \implies \alpha = -\frac{c}{a + b}\\
        (a - b) \beta &= 0
\end{align*}
If $a - b = 0$, then $z = -\frac{c}{a + b} + ti$ for all $t \in \mathbb{R}$, which is a line.

\subsubsection{Exercise 2}
We have the following:
\begin{itemize}
        \item Ellipse: Given foci $f_1, f_2$, we can write the equation as $\abs{z - f_1} + \abs{z - f_2} = 2a$.
        \item Hyperbola: Given foci $f_1, f_2$, we can write the equation as
                $\abs{\abs{z - f_1} - \abs{z - f_2}} = 2a$.
        \item Parabola is a little trickier, since now we're concerned with distance to a point (focus) as
                well as the distance to a line (directrix). One can compute the minimum distance to the
                directrix using a number of methods (calculus, complex inner product, etc.). Out of laziness
                I'm just going to call the distance to the directrix $\abs{z - d}$. Then we have that the
                equation for a parabola looks like $\abs{z - f} = \abs{z - d}$.
\end{itemize}

\subsection{The Spherical Representation}

\subsubsection{Exercise 1}
If the stereographic projections of $z$ and $z'$ are diametrically opposite, the distance between the 
projections is 2. Thus
\begin{align*}
        \frac{\abs{z - z'}^2}{(1 + \abs{z}^2) (1 + \abs{z'}^2} &= 1 \\
        \abs{z}^2 -z\bar{z}' - \bar{z}z' + \abs{z'}^2 &= 1 + \abs{z}^2 + \abs{z'}^2 + \abs{z}^2 \abs{z'}^2 \\
        z\bar{z}' + \bar{z}z' + z\bar{z}' \bar{z}z' &= -1 \\
        (z\bar{z}' + 1)(\bar{z}z' + 1) &= 0 \implies z\bar{z}' = -1
\end{align*}
The other direction follows from reversing the steps above.

\subsubsection{Exercise 2}
Using the fact that the diameter of the sphere is 2, we can solve for the inscribed cube's sidelength by 
applying the Pythagorean theorem to the triangle whose hypotenuse joins the bottom left and top right 
vertices of the cube. This gives us that $2^2 - 2s^2 = s^2 \implies s = \sqrt{\frac{4}{3}}$. Now if we 
label the bottom left corner of the cube as $(x_1, x_2, x_3)$ and the top right corner of the cube as
$(x_1', x_2', x_3')$, we can use the fact that $x_i = -x_i'$ to get that $x_1 = x_2 = x_3 = \sqrt{\frac{1}{3}}$ 
. Thus, the vertices of the cube are $(\pm \sqrt{\frac{1}{3}}, \pm \sqrt{\frac{1}{3}}, \pm \sqrt{\frac{1}{3}})$
from which it is then straightforward to compute the stereographic projections.
