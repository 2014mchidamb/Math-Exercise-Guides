\section{Functions on Euclidean Space}
\textbf{NOTE:} My notes differ from Spivak's text in that I use subscripts to denote
components instead of superscripts.

\subsection{Norm and Inner Product}

\subsubsection{Exercise 1}
\begin{align*}
        \sum_{i = 1}^n x_i^2  = \sum_{i = 1}^n \abs{x_i}^2 \leq \bigg(\sum_{i = 1}^n \abs{x_i}\bigg)^2 
        \implies \sqrt{\sum_{i = 1}^n x_i^2} \leq \sum_{i = 1}^n \abs{x_i}^2 
\end{align*}

\subsubsection{Exercise 2}
We need $\sum_{i = 1}^n x_i y_i = \abs{x} \abs{y}$. Linear dependence by itself is not enough, since if
$x$ and $y$ are opposite sign we see that the lefthand sum will be negative. Thus, we need linear dependence 
as well as $x$ and $y$ having the same sign.

\subsubsection{Exercise 3}
The proof is identical to the $\abs{x + y}$ case, except now we have a $-2 \sum_{i = 1}^n x_i y_i$ term.
Thus, equality holds when $x$ and $y$ are linearly dependent and have opposite signs.

\subsubsection{Exercise 4}
$\abs{\abs{x} - \abs{y}}^2 = \abs{x}^2 + \abs{y}^2 - 2\abs{x} \abs{y}$. Since $\abs{x} \abs{y} \geq \langle x, y \rangle$, we have the desired inequality.

\subsubsection{Exercise 5}
\begin{align*}
        \abs{z - x} = \abs{z - y + y - x} \leq \abs{z - y} + \abs{y - x}
\end{align*}
Geometrically, this is just the fact that any sidelength of a triangle must be bounded by the sum of the
other two sidelengths.

\subsubsection{Exercise 6}
(a) We can proceed as Spivak hints by noting that for the $\int_{a}^{b} (f - \lambda g)^2 > 0$ case, the
proof is identical to that of Theorem 1-1 (2). For the $\int_{a}^{b} (f - \lambda g)^2 = 0$ case (which is when equality is obtained), we
can use the fact that $(f - \lambda g)^2 = 0$ almost everywhere.

However, I think it's a little smoother to handle both cases at once:
\begin{align*}
        \int_{a}^{b} (f - \lambda g)^2 = \int_{a}^{b} f^2 - 2\lambda \int_{a}^{b} fg + \lambda^2 \int_{a}^{b} g^2 &\geq 0 \\ 
        \int_{a}^{b} f^2 - \frac{2 \bigg(\int_{a}^{b} fg\bigg)^2}{\int_{a}^{b} g^2} + \frac{\bigg(\int_{a}^{b} fg\bigg)^2}{\int_{a}^{b} g^2} &\geq 0 \quad \text{for} \: \: \lambda = \frac{\int_{a}^{b} fg}{\int_{a}^{b} g^2} \\
        \sqrt{\int_{a}^{b} f^2 \int_{a}^{b} g^2} &\geq \abs{\int_{a}^{b} fg}
\end{align*}

(b) Equality does not necessarily imply that $f = \lambda g$, since we could consider $f$ and $g$ to be
0 everywhere except two points $a$ and $b$ such that $f(a) \neq \lambda g(a)$ and $f(b) \neq \lambda g(b)$.
However, if $f$ and $g$ are continuous, then $\int_{a}^{b} (f - \lambda g)^2 = 0$ implies that $f = \lambda g$.

(c) Define $f(m) = x_i$ and $g(m) = y_i$ for $m \in [i, i + 1)$. Then $\int_{1}^{n} f g = \sum_{1}^n x_i y_i$ 
(we can break up the integral at the points of discontinuity) and Theorem 1-1 (2) follows.

\subsubsection{Exercise 7}
(a) If $T$ is inner product preserving then we have $\langle Tx, Tx \rangle = \langle x, x \rangle$, so
$T$ is norm preserving. If $T$ is norm preserving, then
\begin{align*}
        \langle T(x - y), T(x - y) \rangle &= \abs{T(x)}^2 + \abs{T(y)}^2 - 2\langle T(x), T(y) \rangle \\
        \langle x - y, x - y \rangle &= \abs{x}^2 + \abs{y}^2 - 2 \langle x, y \rangle \\
        \implies \langle T(x), T(y) \rangle &= \langle x, y \rangle
\end{align*}
so $T$ is inner product preserving.

(b) Since $Tx = Ty \implies T(x - y) = 0 \implies \abs{x - y} = 0$, $T$ is injective. Furthermore,  
$Tx = 0 \implies \abs{x} = 0$, so the nullspace of $T$ is trivial and $T$ is thus surjective. Now if we
consider $T^{-1} y = x$, then we have $\langle y, y \rangle = \langle TT^{-1} y, TT^{-1} y \rangle = \langle T^{-1} y, T^{-1} y \rangle$.

\subsubsection{Exercise 10}
Let $\norm{T}_F = \sqrt{\sum_{i = 1}^n \sum_{j = 1}^m T_{ij}^2}$ (Frobenius norm). Then we have
\begin{align*}
        \abs{T h}^2 = \sum_{i = 1}^n \bigg(\sum_{j = 1}^m T_{ij} h_j\bigg)^2 &\leq \sum_{i = 1}^n \sum_{j = 1}^m T_ij^2 \sum_{j = 1}^m h_j^2 = \norm{T}_F^2 \abs{h}^2
\end{align*}
so letting $M = \norm{T}_F$ gives the desired inequality.

\subsubsection{Exercise 12}
Linearity and injectivity of $T$ follow from linearity of inner product. To see surjectivity, we note that 
any element $f \in (\mathbb{R}^n)^*$ is determined entirely by $f(e_1), ..., f(e_n)$ due to linearity. Thus,
$f(y) = \langle x, y \rangle$ for the unique $x$ satisfying $x_i = f(e_i)$.

\subsubsection{Exercise 13}
Expanding $\langle x + y, x + y \rangle$ gives the desired result.

\subsection{Subsets of Euclidean Space}

\subsubsection{Exercise 14}
Any point $a$ in the union is also in some set which contains an open set around $a$, and by the definition of
union this open set is also in the union. For finite intersection, if a point is in two open sets, then both
sets must contain some open rectangles around that point; the smaller of these rectangles is in the 
intersection of both sets. For infinite intersection, we can consider the intersection of 
$(-\frac{1}{n}, \frac{1}{n})$ for all $n \in \mathbb{N}$, as it consists only of the single point 0.

\subsubsection{Exercise 17}
The procedure hinted by Spivak seems to be to split the square into 4 quadrants and then select a point from
each quadrant while satisfying the constraint, then repeat this procedure indefinitely (split the 4 quadrants
into 16 quadrants, etc.). However, I'm not sure how to make this procedure more rigorous.

\subsubsection{Exercise 19}
We can use the facts that the rationals are dense in $\mathbb{R}$ and that closed sets contain all of their
limit points (although neither fact is proved so far in this book) to immediately get the desired result.

\subsubsection{Exercise 21}
(a) Since $A$ is closed, $A^c$ is open, which means $x$ has an open rectangle (and thus, an open ball) around
it that is contained within $A^c$. Therefore, there exists $d > 0$ such that $\abs{y - x} \geq d$ for all
$y \in A$.

(b) Suppose that for every $d > 0$, we could choose $y_i \in A$ and $x_i \in B$ such that $\abs{y_i - x_i} < d$.
This would imply that we could pick a sequence $\{y_n\} \in A$ and a sequence $\{x_n\} \in B$ such that
 $\lim_{n \to \infty} y_n = \lim_{n \to \infty} x_n$. Since $B$ is compact, this limit must be a point in
  $B$, which contradicts the disjointness of  $A$ and  $B$.

(c) Consider $A = \{(0, i) \: | \: i \in \mathbb{N}\}$ and $B = \{(0, i + \frac{1}{i}) \: | \: i \in \mathbb{N}\}$.
For any $d$, we can choose $i$  such that $\frac{1}{i} < d$.

\subsubsection{Exercise 22}
This is easier to argue with open/closed balls instead of open/closed rectangles (for me). Every point
$c \in C$ must have an open ball $B_{\delta} (c) \subset U$ for some $\delta > 0$. For each such ball,
consider instead $B_{r} (c)$ with $r < \delta$. The closure of $B_r(c)$ is a closed ball that is contained
within $U$. Now, since $C$ is compact, it can be covered by finitely many of these closed balls. Since
closed sets are closed under finite union, we can take $D$ to be the union of these balls to get a compact
set whose interior contains $C$.

\subsection{Functions and Continuity}

\subsubsection{Exercise 23}
Suppose $\lim_{x \to a} f(x) = b$. Then we can choose $\delta$ such that $\abs{x - a} < \delta$ implies
$\abs{f(x) - b} < \epsilon$ for any $\epsilon > 0$. Rewriting $\abs{f(x) - b}$ as 
$\sqrt{\sum_{i = 1}^m (f_i (x) - b_i)^2}$ gives $\abs{f_i(x) - b_i} < \epsilon$ for the same $\delta$, so
we get $\lim_{x \to a} f_i(x) = b_i$. The other direction reverses the steps after choosing 
$\abs{f_i(x) - b_i} < \frac{\epsilon}{\sqrt{m}}$. 

\subsubsection{Exercise 24}
This is basically the same as the previous exercise.

\subsubsection{Exercise 25}
From Exercise 1-10 we have that there exists  $M$  such that $\abs{T(x)} \leq M \abs{x}$. Thus, for any
$\epsilon > 0$ and $a \in \mathbb{R}^n$, we can choose $\delta = \frac{\epsilon}{M}$. Using this $\delta$,
we get $\abs{x - a} < \delta$ implies $\abs{T(x) - T(a)} = \abs{T(x - a)} < \epsilon$, so $T$ is continuous.

\subsubsection{Exercise 28}
Choose any $x$ on the boundary of  $A$ and let  $f(y) = \frac{1}{\abs{y - x}}$. Since $f$ is the quotient
of two continuous functions ($g(y) = 1$ and  $h(y) = \abs{y - x}$) with non-zero denonimator, $f$ is continuous.
Furthermore, by construction we can choose  $y$ arbitrarily close to  $x$, so  $f$ is unbounded.

\subsubsection{Exercise 29}
Since $f$ is continuous,  $f(A)$ is compact. By Heine-Borel,  $f(A) \subset \mathbb{R}$ is closed and bounded,
so it contains its maximum and minimum.

\subsubsection{Exercise 30}
Let $x_0 = a$. We have that
\begin{align*}
        \sum_{i = 1}^n o(f, x_i) &= \lim_{\delta \to 0} \sum_{i = 1}^n [M(f, x_i, \delta) - m(f, x_i, \delta)] \\
                                 &\leq \lim_{\delta \to 0} \sum_{i = 1}^n [M(f, x_i, \delta) - M(f, x_{i - 1}, \delta)] \\
                                 &\leq \lim_{\delta \to 0} M(f, x_n, \delta) - m(f, a, \delta) \\
                                 &\leq f(b) - f(a)
\end{align*}
Where we used the fact that $f$ is increasing to get $m(f, x_i, \delta) \geq M(f, x_{i - 1}, \delta)$.
