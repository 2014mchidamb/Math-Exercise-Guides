\section{Differentiation}

\subsection{Basic Definitions}

\subsubsection{Exercise 1}
Let the derivative of $f$ at $a$ be the linear operator $\lambda$. Then, by problem $1-10$, we have that
$\abs{\lambda(h)} \leq M \abs{h}$ for some $M \in \mathbb{R}$. Thus,
\begin{align*}
        \abs{(f(a + h) - f(a)} &= \frac{\abs{f(a + h) - f(a)}}{\abs{h}} * \abs{h} \\
                               &\leq \frac{\abs{f(a + h) - f(a) - \lambda(h)} + \abs{\lambda(h)}}{\abs{h}} * \abs{h} \\
        \implies \lim_{h \to 0} \abs{f(a + h) - f(a)} &\leq \lim_{h \to 0} \frac{\abs{f(a + h) - f(a) - \lambda(h)} + \abs{\lambda(h)}}{\abs{h}} \lim_{h \to 0} \abs{h} \\
                                                      &\leq M * 0 = 0
\end{align*}

\subsubsection{Exercise 2}
If $f$ is independent, we can define $g = f(x, y_0)$ for some arbitrary $y_0$. If there exists $g$ such that
$f(x, y) = g(x)$ for all $x, y$, then we have $f(x, y_1) = g(x) = f(x, y_2)$ so $f$ is independent of the
second variable. In this case, $f'(a, b) = g'(a)$.

\subsubsection{Exercise 3}
A function that is independent of both variables must, by construction, be constant.

\subsubsection{Exercise 4}
(a) When $t < 0$, we have
\begin{align*}
        h(t) = -t\abs{x} g\bigg(\frac{tx}{-t \abs{x}}\bigg) = t\abs{x}g\bigg(\frac{x}{\abs{x}}\bigg)
\end{align*}
which is the same as $h(t)$ when $t > 0$, so $h(t)$ is differentiable with derivative $f(x)$.

(b) Since $g(0, 1) = g(1, 0) = 0$, we have that $f(h, 0) = f(0, k) = 0$. Letting $\lambda = Df(0, 0)$, we have
that
\begin{align*}
        \lim_{(h, 0) \to 0} \frac{\abs{\lambda(h, 0)}}{\abs{h}} &= \lim_{(h, 0) \to 0} \frac{\abs{h} \lambda(1, 0)}{\abs{h}} \\
                                                                &= \lambda(1, 0) = 0
\end{align*}
if $\lambda$ exists. Similarly, considering $(0, k) \to 0$ gives $\lambda(0, 1) = 0$, so $\lambda = 0$. 
As a result, we have that $f$ is differentiable at $(0, 0)$ only if
\begin{align*}
        \lim_{(h, k) \to (0, 0)} \frac{\abs{f(h, k)}}{\abs{(h, k)}} = \lim_{(h, k) \to (0, 0)} \abs{g\bigg(\frac{(h, k)}{\abs{(h, k)}}\bigg)} = 0
\end{align*}
which implies that $g = 0$ (we can consider $t(x, y)$ as $t \to 0$ for every $(x, y)$ on the unit circle). 

\subsubsection{Exercise 5}
Apply the previous exercise with $g(x, y) = x\abs{y}$.

\subsubsection{Exercise 8}
If $f$ is differentiable, then
\begin{align*}
        \lim_{h \to 0} \frac{\abs{f(a + h) - f(a) - \lambda(h)}}{\abs{h}} &= 0 \\
                                                                          &= \lim_{h \to 0} \frac{\abs{(f_1(a + h) - f_1(a) - \lambda_1(h), f_2(a + h) - f_2(a) - \lambda_2(h))}}{\abs{h}} \\
                                                                          &\geq \lim_{h \to 0} \frac{\abs{f_1(a + h) - f_1(a) - \lambda_1(h)}}{\abs{h}}
\end{align*}
so $f_1$ and $f_2$ are both differentiable. The reverse direction is a straightforward application of the triangle
inequality.

\subsubsection{Exercise 9}
(a) If $f$ is differentiable at $a$, then we can let $g(x) = f(a) + f'(a) (x - a)$. For the other direction,
\begin{align*}
        \lim_{h \to 0} \frac{f(a + h) - a_0 - a_1h}{h} &= 0 \\
        \implies \lim_{h \to 0} \frac{f(a + h) - a_0}{h} &= a_1 \\
        \implies a_0 = f(a)
\end{align*}
so $f$ is differentiable at $a$.

(b) We can break up the $n^{\text{th}}$ order limit into
\begin{align*}
        \lim_{x \to  a} \frac{f(x) - g(x)}{(x - a)^n} &=
        \lim_{x \to a} \frac{f(x) - \sum_{i = 0}^{n - 1} \frac{f^{(i)}(a)}{i!} (x - a)^i}{(x - a)^n} +
        \frac{f^{(n)}(a)}{n!} \\
                                                      &= \frac{f^{(n)}(a)}{n!} - \frac{f^{(n)}(a)}{n!} = 0
\end{align*}
where the last step follows from repeated application of L'Hopital's.

\subsection{Basic Theorems}

\subsubsection{Exercise 11}
(a) Let $G$ be a function such that $G' = g$. Then
\begin{align*}
        f'(x, y) &= g(x + y) (\pi_1'(x, y) + \pi_2'(x, y)) \\
                 &= g(x + y) ((1, 0) + (0, 1)) = (g(x + y), g(x + y))
\end{align*}

(b) Following the setup of (a), we have
\begin{align*}
        f'(x, y) &= g(xy) (y \pi_1'(x, y) + x \pi_2'(x, y)) \\
                 &= g(xy) (y, x)
\end{align*}

\subsubsection{Exercise 12}
(a) Since $f(h, k) = hk f(1, 1)$, the desired result follows immediately from $\lim_{(h, k) \to 0} \frac{\abs{hk}}{\abs{(h, k)}} = 0$.

(b) We have that $f(a + h, b + k) - f(a, b) = f(a, k) + f(h, b) + f(h, k)$ from bilinearity, so $Df(a, b)(x, y) = f(a, y) + f(x, b)$ as
desired.

(c) The function $p: \mathbb{R} \times \mathbb{R} \to \mathbb{R}$ from Theorem 2-3 is bilinear, so it is just a special case
of (b).

\subsubsection{Exercise 14}
(a) We can bound the multilinear case with the bilinear case:
\begin{align*}
        \lim_{h \to 0} \frac{\abs{f(a_1, ..., h_i, ..., h_j, ..., a_k)}}{\abs{h}} &\leq
        \lim_{h \to 0} \frac{\abs{f(a_1, ..., h_i, ..., h_j, ..., a_k)}}{\abs{(h_i, h_j)}} \\ &\leq
        \lim_{(h_i, h_j) \to 0} \frac{\abs{f(a_1, ..., h_i, ..., h_j, ..., a_k)}}{\abs{(h_i, h_j)}} \\
                                                                                             &= 0
\end{align*}

(b) Expanding $f(a_1 + h_1, ..., a_n + h_n) - f(a_1, ..., a_n) - Df(a_1, ..., a_n)(h_1, ..., h_n)$
leaves only terms that are at least bilinear, so by part (a) we are done.

\subsubsection{Exercise 16}
Using the fact that $Dx = x$ (from Theorem 2-3), we have that
\begin{align*}
        f'(f^{-1}) \circ f^{-1\prime} (x) &= x \\
        \implies f^{-1\prime} (x) &= [f'(f^{-1} (x))]^{-1} 
\end{align*}
