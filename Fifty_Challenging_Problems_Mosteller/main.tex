\documentclass{article}
\usepackage[utf8]{inputenc}
\usepackage{amsmath}
\usepackage{amssymb}
\usepackage{framed}
\usepackage{hyperref}
\usepackage[parfill]{parskip}
\usepackage{physics}

\begin{document}
\title{Problem Guide for \textit{Fifty Challenging Problems in Probability with Solutions} by Mosteller}
\author{Muthu Chidambaram}
\date{Last Updated: \today}

\maketitle

\tableofcontents
\newpage 

\section*{About}

\begin{quote}
        \textit{``A man must love a thing very much if he not only practices it without any hope of fame or money, but even practices it without any hope of doing it well.''} 
        - G.K. Chesterton (\href{https://mathoverflow.net/questions/43690/whats-a-mathematician-to-do/44213#44213}{Maybe})
\end{quote}

Almost certainly my most useless set of notes, as the title of this book says \textit{with Solutions}. 

\newpage

\section{The Sock Drawer}
Let the number of socks in the drawer be $n$, and let the number of red socks be $k$. Then we are given that
$\binom{k}{2} / \binom{n}{2}$ is $\frac{1}{2}$. Thus, we can constrain our possibilities as follows
\begin{align*}
        \binom{k}{2} + \binom{k}{1} \binom{n - k}{1} + \binom{n - k}{2} &= \binom{n}{2} \\
        \binom{k}{1} \binom{n - k}{1} + \binom{n - k}{2} &= \frac{1}{2} \binom{n}{2} \\
        n(n - 1) &= 2k(k - 1)
\end{align*}
From the last equation, we have that $n = 4, k = 3$ is one solution. Additionally, we can note that
 $21 * 20 = 3 * 7 * 5 * 2^2 = 2 * 15 * 14$ to get a solution with an even number of black socks. 
 Beyond that, I'll have to get back to you; I haven't started reading number theory books yet.

\section{Successive Wins}
Let the probability that Elmer beats his dad be $p$ and let the probability that he beats the champion be $q$.
We assume that these two probabilities are independent (which may not be safe, hot streaks are a thing).
We are given that  $p > q$. The probability that Elmer wins the prize in the dad-champ-dad series is
$p^2q + 2pq(1-p) = 2pq - p^2q$ (either Elmer wins all 3, the first 2, or the last 2). Similarly, the 
probability for the other series is  $2pq - pq^2$. Since $p^2q > pq^2$ from our assumption that $p > q$,
Elmer should choose the champ-dad-champ series.

\section{The Flippant Juror}
We see that the probability that the three-man jury succeeds is $p(1-p) + p^2 = p$, since either both 
non-coin-flipping jurors pick correctly or one of them picks incorrectly but the coin flip is correct.
Thus, the one-man and three-man juries both have the same probability of being correct.

\section{Trials until First Success}
This question is the same as asking for the expected value of a geometric random variable with $p = \frac{1}{6}$
, so the answer is 6.

\section{Coin in Square}
Consider a table consisting of just a single 1-by-1 square. The probability that the coin lands entirely within
this square is the same as  the probability of picking a point $(x, y)$ with $\frac{3}{8} \leq x, y \leq
\frac{5}{8}$ (assuming the position of the center of the coin is uniformly distributed). 
This is because we can consider the square with opposite corners $(0, 0)$ and $(1, 1)$ and note
that the coin is only within this square if its center is at least $\frac{3}{8}$ (its radius) away from each
of the boundaries. Since these constraints form a square with sidelength $\frac{1}{4}$, the probability the
coin lands in a single 1-by-1 square is $\frac{1}{16}$. Since the squares are all 1-by-1, the total area of the
table scales exactly as the probability, so the number of squares does not matter. Thus, the probability
is just $\frac{1}{16}$.

\end{document}
