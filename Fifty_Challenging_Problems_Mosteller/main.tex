\documentclass{article}
\usepackage[utf8]{inputenc}
\usepackage{amsmath}
\usepackage{amssymb}
\usepackage{framed}
\usepackage{hyperref}
\usepackage[parfill]{parskip}
\usepackage{physics}

\begin{document}
\title{Problem Guide for \textit{Fifty Challenging Problems in Probability with Solutions} by Mosteller}
\author{Muthu Chidambaram}
\date{Last Updated: \today}

\maketitle

\tableofcontents
\newpage 

\section*{About}

\begin{quote}
        \textit{``A man must love a thing very much if he not only practices it without any hope of fame or money, but even practices it without any hope of doing it well.''} 
        - G.K. Chesterton (\href{https://mathoverflow.net/questions/43690/whats-a-mathematician-to-do/44213#44213}{Maybe})
\end{quote}

Almost certainly my most useless set of notes, as the title of this book says \textit{with Solutions}. 

\newpage

\section{The Sock Drawer}
Let the number of socks in the drawer be $n$, and let the number of red socks be $k$. Then we are given that
$\binom{k}{2} / \binom{n}{2}$ is $\frac{1}{2}$. Thus, we can constrain our possibilities as follows
\begin{align*}
        \binom{k}{2} + \binom{k}{1} \binom{n - k}{1} + \binom{n - k}{2} &= \binom{n}{2} \\
        \binom{k}{1} \binom{n - k}{1} + \binom{n - k}{2} &= \frac{1}{2} \binom{n}{2} \\
        n(n - 1) &= 2k(k - 1)
\end{align*}
From the last equation, we have that $n = 4, k = 3$ is one solution. Additionally, we can note that
 $21 * 20 = 3 * 7 * 5 * 2^2 = 2 * 15 * 14$ to get a solution with an even number of black socks. 
 Beyond that, I'll have to get back to you; I haven't started reading number theory books yet.

\end{document}
